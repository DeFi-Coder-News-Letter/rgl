\HeaderA{axes3d}{Draw boxes, axes and other text outside the data}{axes3d}
\aliasA{axis3d}{axes3d}{axis3d}
\aliasA{box3d}{axes3d}{box3d}
\aliasA{mtext3d}{axes3d}{mtext3d}
\aliasA{title3d}{axes3d}{title3d}
\keyword{dynamic}{axes3d}
\begin{Description}\relax
These functions draw axes, boxes and text outside the range of the data.
\code{axes3d}, \code{box3d} and \code{title3d} are the higher level functions; 
normally the others need not be called directly by users.
\end{Description}
\begin{Usage}
\begin{verbatim}
axes3d(edges = "bbox", labels = TRUE, tick = TRUE, nticks = 5, ...)
box3d(...) 
title3d(main = NULL, sub = NULL, xlab = NULL, ylab = NULL, 
    zlab = NULL, line = NA, ...) 
axis3d(edge, at = NULL, labels = TRUE, tick = TRUE, line = 0, 
    pos = NULL, nticks = 5, ...) 
mtext3d(text, edge, line = 0, at = NULL, pos = NA, ...) 
\end{verbatim}
\end{Usage}
\begin{Arguments}
\begin{ldescription}
\item[\code{edges}] a code to describe which edge(s) of the box to use; see Details below 
\item[\code{labels}] whether to label the axes, or (for \code{axis3d}) the
labels to use
\item[\code{tick}] whether to use tick marks 
\item[\code{nticks}] suggested number of ticks 
\item[\code{main}] the main title for the plot 
\item[\code{sub}] the subtitle for the plot 
\item[\code{xlab, ylab, zlab}] the axis labels for the plot 
\item[\code{line}] the ``line'' of the plot margin to draw the label on 
\item[\code{edge, pos}] the position at which to draw the axis or text 
\item[\code{text}] the text to draw 
\item[\code{at}] the value of a coordinate at which to draw the axis 
\item[\code{...}] additional parameters which are passed to \code{\LinkA{bbox3d}{bbox3d}} or \code{\LinkA{material3d}{material3d}} 
\end{ldescription}
\end{Arguments}
\begin{Details}\relax
The rectangular prism holding the 3D plot has 12 edges.  They are identified
using 3 character strings.  The first character (`x', `y', or `z') selects 
the direction of the axis.  The next two characters are each `-' or `+',
selecting the lower or upper end of one of the other coordinates.  If only
one or two characters are given, the remaining characters default to `-'.  
For example \code{edge = 'x+'} draws an x-axis at the high level of y and the
low level of z.

By default, \code{axes3d} uses the \code{\LinkA{bbox3d}{bbox3d}} function to draw the axes.  
The labels will move so that they do not obscure the data.  Alternatively,
a vector of arguments as described above may be used, in which case
fixed axes are drawn using \code{axis3d}.

If \code{pos} is a numeric vector of length 3, \code{edge} determines
the direction of the axis and the tick marks, and the values of the
other two coordinates in \code{pos} determine the position.  See the
examples.
\end{Details}
\begin{Value}
These functions are called for their side effects.  They return the object IDs of
objects added to the scene.
\end{Value}
\begin{Author}\relax
Duncan Murdoch
\end{Author}
\begin{SeeAlso}\relax
\code{\LinkA{axis}{axis}}, \code{\LinkA{box}{box}},
\code{title}, \code{mtext}, \LinkA{bbox3d}{bbox3d}
\end{SeeAlso}
\begin{Examples}
\begin{ExampleCode}
  open3d()
  points3d(rnorm(10),rnorm(10),rnorm(10), size=3)

  # First add standard axes
  axes3d()  

  # and one in the middle (the NA will be ignored, a number would 
  # do as well)
  axis3d('x',pos=c(NA, 0, 0))

  # add titles
  title3d('main','sub','xlab','ylab','zlab')

  rgl.bringtotop()
  
  open3d()
  points3d(rnorm(10),rnorm(10),rnorm(10), size=3)
  
  # Use fixed axes
  
  axes3d(c('x','y','z'))
         
  # Put 4 x-axes on the plot
  axes3d(c('x--','x-+','x+-','x++'))         
  
  axis3d('x',pos=c(NA, 0, 0))     
  title3d('main','sub','xlab','ylab','zlab')
\end{ExampleCode}
\end{Examples}


\HeaderA{rgl.postscript}{export screenshot}{rgl.postscript}
\keyword{dynamic}{rgl.postscript}
\begin{Description}\relax
Saves the screenshot as PostScript file.
\end{Description}
\begin{Usage}
\begin{verbatim}
rgl.postscript( filename, fmt="eps" )
\end{verbatim}
\end{Usage}
\begin{Arguments}
\begin{ldescription}
\item[\code{filename}] full path to filename.
\item[\code{fmt}] export format, currently supported: ps, eps, tex, pdf, svg, pgf 
\end{ldescription}
\end{Arguments}
\begin{Details}\relax
Animations can be created in a loop modifying the scene and saving 
a screenshot to a file. (See example below)
\end{Details}
\begin{Author}\relax
Christophe Geuzaine / Albrecht Gebhardt
\end{Author}
\begin{References}\relax
GL2PS: an OpenGL to PostScript printing library by Christophe Geuzaine
\end{References}
\begin{SeeAlso}\relax
\code{\LinkA{rgl.viewpoint}{rgl.viewpoint}}, \code{\LinkA{rgl.snapshot}{rgl.snapshot}}
\end{SeeAlso}
\begin{Examples}
\begin{ExampleCode}

## Not run: 

#
# create a series of frames for an animation
#

rgl.open()
shade3d(oh3d(), color="red")
rgl.viewpoint(0,20)

for (i in 1:45) {
  rgl.viewpoint(i,20)
  filename <- paste("pic",formatC(i,digits=1,flag="0"),".eps",sep="") 
  rgl.postscript(filename, fmt="eps")
}

## End(Not run)

\end{ExampleCode}
\end{Examples}


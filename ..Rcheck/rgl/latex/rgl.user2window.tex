\HeaderA{rgl.user2window}{Convert between rgl user and window coordinates}{rgl.user2window}
\aliasA{rgl.projection}{rgl.user2window}{rgl.projection}
\aliasA{rgl.window2user}{rgl.user2window}{rgl.window2user}
\keyword{dynamic}{rgl.user2window}
\begin{Description}\relax
This function converts from 3-dimensional user coordinates
to 3-dimensional window coordinates.
\end{Description}
\begin{Usage}
\begin{verbatim}
rgl.user2window(x, y = NULL, z = NULL, projection = rgl.projection())
rgl.window2user(x, y = NULL, z = 0, projection = rgl.projection())
rgl.projection()
\end{verbatim}
\end{Usage}
\begin{Arguments}
\begin{ldescription}
\item[\code{x, y, z}] Input coordinates.  Any reasonable way of defining the
coordinates is acceptable.  See the function \code{\LinkA{xyz.coords}{xyz.coords}}
for details.
\item[\code{projection}] The rgl projection to use 
\end{ldescription}
\end{Arguments}
\begin{Details}\relax
These functions convert between user coordinates and window coordinates.

Window coordinates run from 0 to 1 in X, Y, and Z.  X runs from 0 on the
left to 1 on the right; Y runs from 0 at the bottom to 1 at the top;
Z runs from 0 foremost to 1 in the background.  \code{rgl} does not currently
display vertices plotted outside of this range, but in normal circumstances will automatically resize the
display to show them.  In the example below this has been suppressed.
\end{Details}
\begin{Value}
The coordinate conversion functions produce a matrix with columns corresponding 
to the X, Y, and Z coordinates.

\code{rgl.projection()} returns a list containing the model matrix, projection matrix
and viewport.  See \code{\LinkA{par3d}{par3d}} for more details.
\end{Value}
\begin{Author}\relax
Ming Chen / Duncan Murdoch
\end{Author}
\begin{SeeAlso}\relax
\code{\LinkA{select3d}{select3d}}
\end{SeeAlso}
\begin{Examples}
\begin{ExampleCode}
open3d()
points3d(rnorm(100), rnorm(100), rnorm(100))
if (interactive() || !.Platform$OS=="unix") {
# Calculate a square in the middle of the display and plot it
square <- rgl.window2user(c(0.25, 0.25, 0.75, 0.75, 0.25), 
                          c(0.25, 0.75, 0.75, 0.25, 0.25), 0.5)
par3d(ignoreExtent = TRUE)
lines3d(square)
par3d(ignoreExtent = FALSE)
}
\end{ExampleCode}
\end{Examples}


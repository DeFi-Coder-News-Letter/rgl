\HeaderA{select3d}{Select a rectangle in an RGL scene}{select3d}
\aliasA{rgl.select3d}{select3d}{rgl.select3d}
\keyword{dynamic}{select3d}
\begin{Description}\relax
This function allows the user to use the mouse to
select a region in an RGL scene.
\end{Description}
\begin{Usage}
\begin{verbatim}
rgl.select3d(button = c("left", "middle", "right"))
select3d(...)
\end{verbatim}
\end{Usage}
\begin{Arguments}
\begin{ldescription}
\item[\code{ button }] Which button to use for selection.
\item[\code{ ... }] Button argument to pass to \code{rgl.select3d}
\end{ldescription}
\end{Arguments}
\begin{Details}\relax
This function selects 3-dimensional regions by allowing the
user to use a mouse to draw a rectangle showing
the projection of the region onto the screen.  It returns
a function which tests points for inclusion in the selected region.

If the scene is later moved or rotated, the selected region will 
remain the same, no longer corresponding to a rectangle on the screen.
\end{Details}
\begin{Value}
Returns a function \code{f(x,y,z)} which tests whether each
of the points \code{(x,y,z)} is in the selected region, returning
a logical vector.  This function accepts input in a wide
variety of formats as it uses \code{\LinkA{xyz.coords}{xyz.coords}} 
to interpret its parameters.
\end{Value}
\begin{Author}\relax
Ming Chen / Duncan Murdoch
\end{Author}
\begin{SeeAlso}\relax
\code{\LinkA{locator}{locator}}
\end{SeeAlso}
\begin{Examples}
\begin{ExampleCode}

# Allow the user to select some points, and then redraw them
# in a different color

if (interactive()) {
 x <- rnorm(1000)
 y <- rnorm(1000)
 z <- rnorm(1000)
 open3d()
 points3d(x,y,z,size=2)
 f <- select3d()
 keep <- f(x,y,z)
 rgl.pop()
 points3d(x[keep],y[keep],z[keep],size=2,color='red')
 points3d(x[!keep],y[!keep],z[!keep],size=2)
}
\end{ExampleCode}
\end{Examples}


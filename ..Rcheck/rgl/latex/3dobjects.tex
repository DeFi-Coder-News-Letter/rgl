\HeaderA{points3d}{add primitive set shape}{points3d}
\aliasA{lines3d}{points3d}{lines3d}
\aliasA{quads3d}{points3d}{quads3d}
\aliasA{segments3d}{points3d}{segments3d}
\aliasA{triangles3d}{points3d}{triangles3d}
\keyword{dynamic}{points3d}
\begin{Description}\relax
Adds a shape node to the current scene
\end{Description}
\begin{Usage}
\begin{verbatim}
points3d(x, y = NULL, z = NULL,  ...)
lines3d(x, y = NULL, z = NULL,  ...)
segments3d(x, y = NULL, z = NULL, ...)
triangles3d(x, y = NULL, z = NULL, ...)
quads3d(x, y = NULL, z = NULL, ...)
\end{verbatim}
\end{Usage}
\begin{Arguments}
\begin{ldescription}
\item[\code{x, y, z}] coordinates. Any reasonable way of defining the
coordinates is acceptable.  See the function \code{\LinkA{xyz.coords}{xyz.coords}}
for details.
\item[\code{ ... }] Material properties. See \code{\LinkA{rgl.material}{rgl.material}} for details.
\end{ldescription}
\end{Arguments}
\begin{Details}\relax
The functions \code{points3d}, \code{lines3d}, \code{segments3d},
\code{triangles3d} and \code{quads3d} add points, joined lines, line segments,
filled triangles or quadrilaterals to the plots.  They correspond to the OpenGL types
\code{GL\_POINTS, GL\_LINE\_STRIP, GL\_LINES, GL\_TRIANGLES} and \code{GL\_QUADS} respectively.  

Points are taken in pairs by \code{segments3d}, triplets as the vertices
of the triangles, and quadruplets for the quadrilaterals.  Colours are applied vertex by vertex; 
if different at each end of a line segment, or each vertex of a polygon, the colours
are blended over the extent of the object.  Quadrilaterals must be entirely 
in one plane and convex, or the results are undefined.

These functions call the lower level functions \code{\LinkA{rgl.points}{rgl.points}}, \code{\LinkA{rgl.linestrips}{rgl.linestrips}},
and so on, and are provided for convenience.

The appearance of the new objects are defined by the material properties.
See \code{\LinkA{rgl.material}{rgl.material}} for details.
\end{Details}
\begin{Value}
Each function returns the integer object ID of the shape that
was added to the scene.  These can be passed to \code{\LinkA{rgl.pop}{rgl.pop}}
to remove the object from the scene.
\end{Value}
\begin{Author}\relax
Ming Chen and Duncan Murdoch
\end{Author}
\begin{Examples}
\begin{ExampleCode}
# Show 12 random vertices in various ways. 

M <- matrix(rnorm(36), 3, 12, dimnames=list(c('x','y','z'), 
                                       rep(LETTERS[1:4], 3)))

# Force 4-tuples to be convex in planes so that quads3d works.

for (i in c(1,5,9)) {
    quad <- as.data.frame(M[,i+0:3])
    coeffs <- runif(2,0,3)
    if (mean(coeffs) < 1) coeffs <- coeffs + 1 - mean(coeffs)
    quad$C <- with(quad, coeffs[1]*(B-A) + coeffs[2]*(D-A) + A)
    M[,i+0:3] <- as.matrix(quad)
}

open3d()

# Rows of M are x, y, z coords; transpose to plot

M <- t(M)
shift <- matrix(c(-3,3,0), 12, 3, byrow=TRUE)

points3d(M, size=2)
lines3d(M + shift)
segments3d(M + 2*shift)
triangles3d(M + 3*shift, col='red')
quads3d(M + 4*shift, col='green')  
text3d(M + 5*shift, texts=1:12)

# Add labels

shift <- outer(0:5, shift[1,])
shift[,1] <- shift[,1] + 3
text3d(shift, 
       texts = c('points3d','lines3d','segments3d',
         'triangles3d', 'quads3d','text3d'),
       adj = 0)
 rgl.bringtotop()
\end{ExampleCode}
\end{Examples}


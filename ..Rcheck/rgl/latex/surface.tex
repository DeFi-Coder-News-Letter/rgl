\HeaderA{rgl.surface}{add height-field surface shape}{rgl.surface}
\keyword{dynamic}{rgl.surface}
\begin{Description}\relax
Adds a surface to the current scene. The surface is defined by 
a matrix defining the height of each grid point and two vectors
defining the grid.
\end{Description}
\begin{Usage}
\begin{verbatim}
rgl.surface(x, z, y, coords=1:3, ...)
\end{verbatim}
\end{Usage}
\begin{Arguments}
\begin{ldescription}
\item[\code{ x }] values corresponding to rows of \code{y}

\item[\code{ y }] matrix of height values

\item[\code{ z }] values corresponding to columns of \code{y}

\item[\code{ coords }] See details

\item[\code{ ... }] Material and texture properties. See \code{\LinkA{rgl.material}{rgl.material}} for details.
\end{ldescription}
\end{Arguments}
\begin{Details}\relax
Adds a surface mesh to the current scene. The surface is defined by 
the matrix of height values in \code{y}, with rows corresponding 
to the values in \code{x} and columns corresponding to the values in 
\code{z}.

The \code{coords} parameter can be used to change the geometric
interpretation of \code{x}, \code{y}, and \code{z}.  The first entry 
of \code{coords} indicates which coordinate (\code{1=X}, 
\code{2=Y}, \code{3=Z}) corresponds to the \code{x} parameter.
Similarly the second entry corresponds to the \code{y} parameter,
and the third entry to the \code{z} parameter.  In this way 
surfaces may be defined over any coordinate plane.

\code{rgl.surface} always draws the surface with the `front' upwards
(i.e. towards higher \code{y} values).  This can be used to render
the top and bottom differently; see \code{\LinkA{rgl.material}{rgl.material}} and
the example below.

\code{NA} values in the height matrix are not drawn.
\end{Details}
\begin{Value}
The object ID of the displayed surface is returned invisibly.
\end{Value}
\begin{SeeAlso}\relax
\code{\LinkA{rgl.material}{rgl.material}}, \code{\LinkA{surface3d}{surface3d}}, \code{\LinkA{terrain3d}{terrain3d}}
\end{SeeAlso}
\begin{Examples}
\begin{ExampleCode}

#
# volcano example taken from "persp"
#

data(volcano)

y <- 2 * volcano        # Exaggerate the relief

x <- 10 * (1:nrow(y))   # 10 meter spacing (S to N)
z <- 10 * (1:ncol(y))   # 10 meter spacing (E to W)

ylim <- range(y)
ylen <- ylim[2] - ylim[1] + 1

colorlut <- terrain.colors(ylen) # height color lookup table

col <- colorlut[ y-ylim[1]+1 ] # assign colors to heights for each point

rgl.open()
rgl.surface(x, z, y, color=col, back="lines")

\end{ExampleCode}
\end{Examples}


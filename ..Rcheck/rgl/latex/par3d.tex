\HeaderA{par3d}{Set or Query RGL Parameters}{par3d}
\aliasA{open3d}{par3d}{open3d}
\aliasA{r3dDefaults}{par3d}{r3dDefaults}
\keyword{dynamic}{par3d}
\begin{Description}\relax
\code{par3d} can be used to set or query graphical parameters in rgl.
Parameters can be set by specifying them as arguments to \code{par3d} in
\code{tag = value} form, or by passing them as a list of tagged
values.
\end{Description}
\begin{Usage}
\begin{verbatim}
par3d(..., no.readonly = FALSE)

open3d(..., params=get("r3dDefaults", envir=.GlobalEnv))

\end{verbatim}
\end{Usage}
\begin{Arguments}
\begin{ldescription}
\item[\code{...}] arguments in \code{tag = value} form, or a list of tagged
values.  The tags must come from the graphical parameters described
below.
\item[\code{no.readonly}] logical; if \code{TRUE} and there are no other
arguments, only those parameters which can be set by a
subsequent \code{par3d()} call are returned.
\item[\code{params}] a list of graphical parameters
\end{ldescription}
\end{Arguments}
\begin{Details}\relax
Parameters are queried by giving one or more character vectors to
\code{par3d}.

\code{par3d()} (no arguments) or \code{par3d(no.readonly=TRUE)} is used to
get \emph{all} the graphical parameters (as a named list).  

\emph{\bold{R.O.}} indicates \emph{\bold{read-only arguments}}: These
may only be used in queries, i.e., they do \emph{not} set anything.

\code{open3d} opens a new rgl device, and sets the parameters as
requested.  The \code{r3dDefaults} list will be used as default
values for parameters.  As installed this sets the point of view to
'world coordinates' (i.e. x running from left to right, y from front
to back, z from bottom to top), the \code{mouseMode} to
\code{(zAxis, zoom, fov)}, and the field of view to 30 degrees.
Users may create their own variable of that name in the global
environment and it will override the installed one.  If there
is a \code{bg} element in the list or the arguments, it should be
a list of arguments to pass to the \code{\LinkA{bg3d}{bg3d}} function to 
set the background.  A \code{material} element should be a list
of arguments to pass to \code{\LinkA{material3d}{material3d}} to set the material.
\end{Details}
\begin{Value}
When parameters are set, their former values are returned in an
invisible named list.  Such a list can be passed as an argument to
\code{par3d} to restore the parameter values.
Use \code{par3d(no.readonly = TRUE)} for the full list of parameters
that can be restored.

When just one parameter is queried, its value is returned directly.
When two or more parameters are queried, the result is a list of
values, with the list names giving the parameters.

Note the inconsistency: setting one parameter returns a list, but
querying one parameter returns an object.
\end{Value}
\begin{Section}{Parameters}
\describe{
\item[\code{FOV}] real.  The field of view, from 1 to 179 degrees.  This controls
the degree of parallax in the perspective view.
Isometric perspective (which would
correspond to \code{FOV=0}) is not currently possible, but one can
approximate it by specifying \code{par3d(FOV=1)}.
\item[\code{ignoreExtent}] logical.  Set to \code{TRUE} so that subsequently plotted objects
will be ignored in calculating the bounding box of the scene.
\item[\code{modelMatrix}] \emph{\bold{R.O.}}; a 4 by 4 matrix describing the position
of the user data.
\item[\code{mouseMode}] character.  A vector of 3 strings describing
what the 3 mouse buttons do. Partial matching is used.  Possible values for \code{mouseMode} are 
\describe{
\item[\code{"none"}] No action for this button.
\item[\code{"trackball"}] Mouse acts as a virtual trackball, rotating the scene.
\item[\code{"xAxis"}] Similar to \code{"trackball"}, but restricted to X axis rotation.
\item[\code{"yAxis"}] Y axis rotation.
\item[\code{"zAxis"}] Z axis rotation.
\item[\code{"polar"}] Mouse rotates the scene by moving in polar coordinates.
\item[\code{"selecting"}] Mouse is used for selection.  This is not normally set by the
user, but is used internally by the \code{\LinkA{select3d}{select3d}} function.
\item[\code{"zoom"}] Mouse is used to zoom the display.
\item[\code{"fov"}] Mouse changes the field of view of the display.
}

\item[\code{projMatrix}] \emph{\bold{R.O.}}; a 4 by 4 matrix describing the current
projection of the scene.
\item[\code{scale}] real.  A vector of 3 values indicating the amount by which to
rescale each axis before display.  Set by \code{\LinkA{aspect3d}{aspect3d}}.
\item[\code{skipRedraw}] whether to update the display.  Set to \code{FALSE} to suspend
updating while making multiple changes to the scene.  See \code{demo(hist3d)} for an example.
\item[\code{userMatrix}] a 4 by 4 matrix describing user actions to
display the scene.
\item[\code{viewport}] \emph{\bold{R.O.}}; real.  A vector giving the dimensions of the
window in pixels.
\item[\code{zoom}] real. A positive value indicating the current
magnification of the scene.
\item[\code{bbox}] \emph{\bold{R.O.}}; real. A vector of six values indicating the current values of the
bounding box of the scene (xmin, xmax, ymin, ymax, zmin, zmax)
}
\end{Section}
\begin{Section}{note}
The \code{"xAxis"}, \code{"yAxis"} and \code{"zAxis"} mouse modes rotate relative to the coordinate
system of the data, regardless of the current orientation of the scene.
\end{Section}
\begin{Section}{Rendering}
The parameters returned by \code{par3d} are sufficient to determine where rgl would render
a point on the screen.  Given a column vector \code{(x,y,z)}, it performs the equivalent of 
the following operations:
\Enumerate{
\item It converts the point to homogeneous coordinates by appending \code{w=1}, giving
the vector \code{v = (x,y,z,1)}.
\item It obtains the model matrix \code{M = par3d("modelMatrix")} and multiplies
the point by this giving \code{u = M \%*\% v}.  Using this location and information on the
normals (which have been similarly transformed), it performs lighting calculations.  (The 
\code{par3d("userMatrix")} is used in the construction of the model matrix, along with
other transformations to center the object being displayed.)
\item It obtains the projection matrix \code{P = par3d("projMatrix")} and
multiplies the point by it giving \code{P \%*\% u = (x2, y2, z2, w2)}.
\item It converts back to Euclidean coordinates by dividing the first 3 coordinates by \code{w2}.
\item The new value \code{z2/w2} represents the depth into the scene of the point.  Depending
on what has already been plotted, this depth might be obscured, in which case nothing more is plotted.
\item If the point is not culled due to depth, the \code{x2} and \code{y2} values are used
to determine the point in the image.  The \code{par3d("viewport")} values are used to translate
from the range \code{(-1, 1)} to pixel locations, and the point is plotted.
}
See ?\LinkA{matrices}{matrices} for more information on homogeneous and Euclidean coordinates.
\end{Section}
\begin{References}\relax
OpenGL Architecture Review Board (1997).  OpenGL Programming Guide.  Addison-Wesley.
\end{References}
\begin{SeeAlso}\relax
\code{\LinkA{rgl.viewpoint}{rgl.viewpoint}} to set \code{FOV} and \code{zoom}.
\end{SeeAlso}
\begin{Examples}
\begin{ExampleCode}
    r3dDefaults
    open3d()
    shade3d(cube3d(color=rep(rainbow(6),rep(4,6))))
    save <- par3d(userMatrix = rotationMatrix(90*pi/180, 1,0,0))
    save
    par3d("userMatrix")    
    par3d(save)
    par3d("userMatrix")
\end{ExampleCode}
\end{Examples}


\HeaderA{viewpoint}{Set up viewpoint}{viewpoint}
\aliasA{rgl.viewpoint}{viewpoint}{rgl.viewpoint}
\aliasA{view3d}{viewpoint}{view3d}
\keyword{dynamic}{viewpoint}
\begin{Description}\relax
Set the viewpoint orientation.
\end{Description}
\begin{Usage}
\begin{verbatim}
view3d( theta = 0, phi = 15, ...)
rgl.viewpoint( theta = 0, phi = 15, fov = 60, zoom = 1, scale = par3d("scale"), 
               interactive = TRUE, userMatrix )
\end{verbatim}
\end{Usage}
\begin{Arguments}
\begin{ldescription}
\item[\code{theta,phi}] polar coordinates
\item[\code{...}] additional parameters to pass to \code{rgl.viewpoint}
\item[\code{fov}] field-of-view angle
\item[\code{zoom}] zoom factor
\item[\code{scale}] real length 3 vector specifying the rescaling to apply to each axis
\item[\code{interactive}] logical, specifying if interactive navigation is allowed
\item[\code{userMatrix}] 4x4 matrix specifying user point of view
\end{ldescription}
\end{Arguments}
\begin{Details}\relax
The viewpoint can be set in an orbit around the data model, using the polar coordinates \code{\\theta}
and \code{phi}.  Alternatively, it can be set in a completely general way using the 4x4 matrix
\code{userMatrix}.  If \code{userMatrix} is specified, \code{theta} and \code{phi} are ignored.

The pointing device of your graphics user-interface can also be used to 
set the viewpoint interactively. With the pointing device the buttons are by default set as follows:

\Itemize{
\item[left] adjust viewpoint position
\item[middle] adjust field of view angle
\item[right or wheel] adjust zoom factor
}
\end{Details}
\begin{SeeAlso}\relax
\code{\LinkA{par3d}{par3d}}
\end{SeeAlso}
\begin{Examples}
\begin{ExampleCode}

# animated round trip tour for 10 seconds

rgl.open()
shade3d(oh3d(), color="red")

start <- proc.time()[3]
while ((i <- 36*(proc.time()[3]-start)) < 360) {
  rgl.viewpoint(i,i/4); 
}

\end{ExampleCode}
\end{Examples}


\HeaderA{r3d}{Generic 3D interface}{r3d}
\keyword{dynamic}{r3d}
\begin{Description}\relax
Generic 3D interface for 3D rendering and computational geometry.
\end{Description}
\begin{Details}\relax
R3d is a design for an interface for 3d rendering and computation without dependency
on a specific rendering implementation. R3d includes a collection
of 3D objects and geometry algorithms.
All r3d interface functions are named \code{*3d}.  They represent generic functions that delegate 
to implementation functions.

The interface can be grouped into 8 categories: Scene Management, Primitive Shapes,
High-level Shapes, Geometry Objects, Visualization, Interaction, Transformation,
Subdivision.  

The rendering interface gives an abstraction to the underlying rendering model. It can
be grouped into four categories:    
\Enumerate{
\item[Scene Management:] A 3D scene consists of shapes, lights and background environment. 
\item[Primitive Shapes:] Generic primitive 3D graphics shapes such as points, lines, triangles, quadrangles and texts. 
\item[High-level Shapes:] Generic high-level 3D graphics shapes such as spheres, sprites and terrain.
\item[Interaction:] Generic interface to select points in 3D space using the pointer device.
}

In this package we include an implementation of r3d using the underlying \code{rgl.*} functions.

3D computation is supported through the use of object structures that live entirely in R.
\Enumerate{
\item[Geometry Objects:] Geometry and mesh objects allow to define high-level geometry for computational purpose such as quadrangle meshes (qmesh3d).
\item[Transformation:] Generic interface to transform 3d objects.
\item[Visualization:] Generic rendering of 3d objects such as dotted, wired or shaded.
\item[Computation:] Generic subdivision of 3d objects.
}

At present, there are two main practical differences between the r3d functions
and the \code{rgl.*} functions is that the r3d functions call
\code{\LinkA{open3d}{open3d}} if there is no device open, and the
\code{rgl.*} functions call \code{\LinkA{rgl.open}{rgl.open}}. By default
\code{\LinkA{open3d}{open3d}} sets the initial orientation of the coordinate
system in 'world coordinates', i.e. a right-handed coordinate system
in which the x-axis increasingfrom left to right, the y-axis
increases with depth into the scene, and the z-axis increases from
bottom to top of the screen.  \code{rgl.*} functions, on the other
hand, use a right-handed coordinate system similar to that used in
OpenGL.  The x-axis matches that of r3d, but the y-axis increases
from bottom to top, and the z-axis decreases with depth into the
scene.  Since the user can manipulate the scene, either system can
be rotated into the other one.  

The r3d functions also preserve the \code{rgl.material} setting across
calls (except for texture elements, in the current implementation), whereas
the \code{rgl.*} functions leave it as set by the last call.

The example code below illustrates the two coordinate systems.
\end{Details}
\begin{SeeAlso}\relax
\code{\LinkA{points3d}{points3d}}
\code{\LinkA{lines3d}{lines3d}}
\code{\LinkA{segments3d}{segments3d}}
\code{\LinkA{triangles3d}{triangles3d}}
\code{\LinkA{quads3d}{quads3d}}
\code{\LinkA{text3d}{text3d}}
\code{\LinkA{spheres3d}{spheres3d}}
\code{\LinkA{sprites3d}{sprites3d}}
\code{\LinkA{terrain3d}{terrain3d}}
\code{\LinkA{select3d}{select3d}}
\code{\LinkA{dot3d}{dot3d}}
\code{\LinkA{wire3d}{wire3d}}
\code{\LinkA{shade3d}{shade3d}}
\code{\LinkA{transform3d}{transform3d}}
\code{\LinkA{rotate3d}{rotate3d}}
\code{\LinkA{subdivision3d}{subdivision3d}}
\code{\LinkA{qmesh3d}{qmesh3d}}
\code{\LinkA{cube3d}{cube3d}}
\code{\LinkA{rgl}{rgl}}
\end{SeeAlso}
\begin{Examples}
\begin{ExampleCode}
    
     x <- c(0,1,0,0)
     y <- c(0,0,1,0)
     z <- c(0,0,0,1)
     labels <- c("Origin", "X", "Y", "Z")
     i <- c(1,2,1,3,1,4)

     rgl.open()
     rgl.texts(x,y,z,labels)
     rgl.texts(1,1,1,"rgl.* coordinates")
     rgl.lines(x[i],y[i],z[i])

     open3d()
     text3d(x,y,z,labels)
     text3d(1,1,1,"*3d coordinates")
     segments3d(x[i],y[i],z[i])
\end{ExampleCode}
\end{Examples}


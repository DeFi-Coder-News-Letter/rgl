\HeaderA{persp3d}{Surface plots}{persp3d}
\methaliasA{persp3d.default}{persp3d}{persp3d.default}
\keyword{dynamic}{persp3d}
\begin{Description}\relax
This function draws plots of surfaces over the
x-y plane. \code{persp3d} is a generic function.
\end{Description}
\begin{Usage}
\begin{verbatim}
persp3d(x, ...)

## Default S3 method:
persp3d(x = seq(0, 1, len = nrow(z)), y = seq(0, 1, len = ncol(z)),
    z, xlim = range(x), ylim = range(y), zlim = range(z, na.rm = TRUE),
    xlab = NULL, ylab = NULL, zlab = NULL, add = FALSE, aspect = !add, ...)
\end{verbatim}
\end{Usage}
\begin{Arguments}
\begin{ldescription}
\item[\code{x, y}] locations of grid lines at which the values in \code{z} are
measured.  These must be in ascending order.  By default, equally
spaced values from 0 to 1 are used.  If \code{x} is a \code{list},
its components \code{x\$x} and \code{x\$y} are used for \code{x}
and \code{y}, respectively.
\item[\code{z}] a matrix containing the values to be plotted.  
Note that \code{x} can be used instead of \code{z} for
convenience.
\item[\code{xlim, ylim, zlim}] x-, y-  and z-limits.  The plot is produced
so that the rectangular volume defined by these limits is visible.
\item[\code{xlab, ylab, zlab}] titles for the axes.  N.B. These must be
character strings; expressions are not accepted.  Numbers will be
coerced to character strings.
\item[\code{add}] whether to add the points to an existing plot.
\item[\code{aspect}] either a logical indicating whether to adjust the aspect ratio, or a new ratio
\item[\code{...}] additional material parameters to be passed to \code{\LinkA{surface3d}{surface3d}}
and \code{\LinkA{decorate3d}{decorate3d}}.
\end{ldescription}
\end{Arguments}
\begin{Details}\relax
This is similar to \code{\LinkA{persp}{persp}} with user interaction.  See \code{\LinkA{plot3d}{plot3d}}
for more details.
\end{Details}
\begin{Value}
This function is called for the side effect of drawing the plot.  A vector 
of shape IDs is returned.
\end{Value}
\begin{Author}\relax
Duncan Murdoch
\end{Author}
\begin{SeeAlso}\relax
\code{\LinkA{plot3d}{plot3d}}, \code{\LinkA{persp}{persp}}
\end{SeeAlso}
\begin{Examples}
\begin{ExampleCode}

# (1) The Obligatory Mathematical surface.
#     Rotated sinc function.

x <- seq(-10, 10, length= 30)
y <- x
f <- function(x,y) { r <- sqrt(x^2+y^2); 10 * sin(r)/r }
z <- outer(x, y, f)
z[is.na(z)] <- 1
open3d()
bg3d("white")
material3d(col="black")
persp3d(x, y, z, aspect=c(1, 1, 0.5), col = "lightblue",
        xlab = "X", ylab = "Y", zlab = "Sinc( r )")

# (2) Add to existing persp plot:

xE <- c(-10,10); xy <- expand.grid(xE, xE)
points3d(xy[,1], xy[,2], 6, col = 2, size = 3)
lines3d(x, y=10, z= 6 + sin(x), col = 3)

phi <- seq(0, 2*pi, len = 201)
r1 <- 7.725 # radius of 2nd maximum
xr <- r1 * cos(phi)
yr <- r1 * sin(phi)
lines3d(xr,yr, f(xr,yr), col = "pink", size = 2)

# (3) Visualizing a simple DEM model

z <- 2 * volcano        # Exaggerate the relief
x <- 10 * (1:nrow(z))   # 10 meter spacing (S to N)
y <- 10 * (1:ncol(z))   # 10 meter spacing (E to W)

open3d()
bg3d("slategray")
material3d(col="black")
persp3d(x, y, z, col = "green3", aspect="iso",
      axes = FALSE, box = FALSE)
\end{ExampleCode}
\end{Examples}


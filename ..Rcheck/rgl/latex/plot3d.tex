\HeaderA{plot3d}{3D Scatterplot}{plot3d}
\aliasA{decorate3d}{plot3d}{decorate3d}
\methaliasA{plot3d.default}{plot3d}{plot3d.default}
\methaliasA{plot3d.qmesh3d}{plot3d}{plot3d.qmesh3d}
\keyword{dynamic}{plot3d}
\begin{Description}\relax
Draws a 3D scatterplot.
\end{Description}
\begin{Usage}
\begin{verbatim}
plot3d(x, ...)
## Default S3 method:
plot3d(x, y, z,  
        xlab, ylab, zlab, type = "p", col,  
        size, radius,
        add = FALSE, aspect = !add, ...)
## S3 method for class 'qmesh3d':
plot3d(x, xlab = "x", ylab = "y", zlab = "z", type = c("shade", "wire", "dots"),
        add = FALSE, ...)       
decorate3d(xlim, ylim, zlim, 
        xlab = "x", ylab = "y", zlab = "z", 
        box = TRUE, axes = TRUE, main = NULL, sub = NULL,
        top = TRUE, aspect = FALSE, ...)
\end{verbatim}
\end{Usage}
\begin{Arguments}
\begin{ldescription}
\item[\code{x, y, z}] vectors of points to be plotted. Any reasonable way of defining the
coordinates is acceptable.  See the function \code{\LinkA{xyz.coords}{xyz.coords}}
for details.
\item[\code{xlab, ylab, zlab}] labels for the coordinates.
\item[\code{type}] For the default method, a single character indicating the type of item to plot.  
Supported types are: 'p' for points, 's' for spheres, 
'l' for lines, 'h' for line segments 
from \code{z=0}, and 'n' for nothing.  For the \code{qmesh3d} method, one of 
'shade', 'wire', or 'dots'.  Partial matching is used.

\item[\code{col}] the colour to be used for plotted items.
\item[\code{size}] the size for plotted items.
\item[\code{radius}] the radius of spheres:  see Details below.
\item[\code{add}] whether to add the points to an existing plot.
\item[\code{aspect}] either a logical indicating whether to adjust the aspect ratio, or a new ratio.
\item[\code{...}] additional parameters which will be passed to \code{\LinkA{par3d}{par3d}}, \code{\LinkA{material3d}{material3d}}
or \code{decorate3d}.
\item[\code{xlim, ylim, zlim}] limits to use for the coordinates.
\item[\code{box, axes}] whether to draw a box and axes.
\item[\code{main, sub}] main title and subtitle.
\item[\code{top}] whether to bring the window to the top when done.
\end{ldescription}
\end{Arguments}
\begin{Details}\relax
\code{plot3d} is a partial 3D analogue of plot.default.

Note that since \code{rgl} does not currently support
clipping, all points will be plotted, and \code{xlim}, \code{ylim}, and \code{zlim}
will only be used to increase the respective ranges.

Missing values in the data are skipped, as in standard graphics.

If \code{aspect} is \code{TRUE}, aspect ratios of \code{c(1,1,1)} are passed to
\code{\LinkA{aspect3d}{aspect3d}}.  If \code{FALSE}, no aspect adjustment is done.  In other
cases, the value is passed to \code{\LinkA{aspect3d}{aspect3d}}.

With \code{type = "s"}, spheres are drawn centered at the specified locations.
The radius may be controlled by \code{size} (specifying the size relative
to the plot display, with \code{size=1} giving a radius 
about 1/20 of the plot region) or \code{radius} (specifying it on the data scale
if an isometric aspect ratio is chosen, or on an average scale
if not).
\end{Details}
\begin{Value}
\code{plot3d} is called for the side effect of drawing the plot; a vector
of object IDs is returned.

\code{decorate3d} adds the usual decorations to a plot:  labels, axes, etc.
\end{Value}
\begin{Author}\relax
Duncan Murdoch
\end{Author}
\begin{SeeAlso}\relax
\code{\LinkA{plot.default}{plot.default}},  
\code{\LinkA{open3d}{open3d}}, \code{\LinkA{par3d}{par3d}}.
\end{SeeAlso}
\begin{Examples}
\begin{ExampleCode}
  open3d()
  x <- sort(rnorm(1000))
  y <- rnorm(1000)
  z <- rnorm(1000) + atan2(x,y)
  plot3d(x, y, z, col=rainbow(1000), size=2)
\end{ExampleCode}
\end{Examples}


\HeaderA{surface3d}{add height-field surface shape}{surface3d}
\aliasA{terrain3d}{surface3d}{terrain3d}
\keyword{dynamic}{surface3d}
\begin{Description}\relax
Adds a surface to the current scene. The surface is defined by 
a matrix defining the height of each grid point and two vectors
defining the grid.
\end{Description}
\begin{Usage}
\begin{verbatim}
surface3d(x, y, z, ...)
terrain3d(x, y, z, ...)
\end{verbatim}
\end{Usage}
\begin{Arguments}
\begin{ldescription}
\item[\code{ x }] values corresponding to rows of \code{z}

\item[\code{ y }] values corresponding to the columns of \code{z}

\item[\code{ z }] matrix of heights

\item[\code{ ... }] Material and texture properties. See \code{\LinkA{rgl.material}{rgl.material}} for details.
\end{ldescription}
\end{Arguments}
\begin{Details}\relax
Adds a surface mesh to the current scene. The surface is defined by 
the matrix of height values in \code{z}, with rows corresponding 
to the values in \code{x} and columns corresponding to the values in 
\code{y}.  This is the same parametrization as used in \code{\LinkA{persp}{persp}}.

\code{surface3d} always draws the surface with the `front' upwards
(i.e. towards higher \code{z} values).  This can be used to render
the top and bottom differently; see \code{\LinkA{rgl.material}{rgl.material}} and
the example below.

For more flexibility in defining the surface, use \code{\LinkA{rgl.surface}{rgl.surface}}.

\code{surface3d} and \code{terrain3d} are synonyms.
\end{Details}
\begin{SeeAlso}\relax
\code{\LinkA{rgl.material}{rgl.material}}, \code{\LinkA{rgl.surface}{rgl.surface}}, \code{\LinkA{persp}{persp}}
\end{SeeAlso}
\begin{Examples}
\begin{ExampleCode}

#
# volcano example taken from "persp"
#

data(volcano)

z <- 2 * volcano        # Exaggerate the relief

x <- 10 * (1:nrow(z))   # 10 meter spacing (S to N)
y <- 10 * (1:ncol(z))   # 10 meter spacing (E to W)

zlim <- range(y)
zlen <- zlim[2] - zlim[1] + 1

colorlut <- terrain.colors(zlen) # height color lookup table

col <- colorlut[ z-zlim[1]+1 ] # assign colors to heights for each point

open3d()
surface3d(x, y, z, color=col, back="lines")

\end{ExampleCode}
\end{Examples}


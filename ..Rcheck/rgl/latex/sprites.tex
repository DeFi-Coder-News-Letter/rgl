\HeaderA{sprites}{add sprite set shape}{sprites}
\aliasA{particles3d}{sprites}{particles3d}
\aliasA{rgl.sprites}{sprites}{rgl.sprites}
\aliasA{sprites3d}{sprites}{sprites3d}
\keyword{dynamic}{sprites}
\begin{Description}\relax
Adds a sprite set shape node to the scene.
\end{Description}
\begin{Usage}
\begin{verbatim}
sprites3d(x, y = NULL, z = NULL, radius = 1, ...)
particles3d(x, y = NULL, z = NULL, radius = 1, ...)
rgl.sprites(x, y = NULL, z = NULL, radius = 1, ...)
\end{verbatim}
\end{Usage}
\begin{Arguments}
\begin{ldescription}
\item[\code{ x, y, z }] point coordinates.  Any reasonable way of defining the
coordinates is acceptable.  See the function \code{\LinkA{xyz.coords}{xyz.coords}}
for details.
\item[\code{ radius }] 
\item[\code{ ... }] material properties, texture mapping is supported
\end{ldescription}
\end{Arguments}
\begin{Details}\relax
Sprites are rectangle planes that are directed towards the viewpoint.
Their primary use is for fast (and faked) atmospherical effects, e.g. particles and clouds
using alpha blended textures.
Particles are Sprites using an alpha-blended particle texture giving
the illusion of clouds and gasses.

If any coordinate is \code{NA}, the sprite is not plotted.
\end{Details}
\begin{Value}
These functions are called for the side effect of displaying the sprites.
The shape ID of the displayed object is returned.
\end{Value}
\begin{SeeAlso}\relax
\code{\LinkA{rgl.material}{rgl.material}}
\end{SeeAlso}
\begin{Examples}
\begin{ExampleCode}
open3d()
particles3d( rnorm(100), rnorm(100), rnorm(100), color=rainbow(100) )
# is the same as
sprites3d( rnorm(100), rnorm(100), rnorm(100), color=rainbow(100),
  lit=FALSE, alpha=.2,
  textype="alpha", texture=system.file("textures/particle.png", package="rgl") )
\end{ExampleCode}
\end{Examples}


\HeaderA{aspect3d}{Set the aspect ratios of the current plot}{aspect3d}
\keyword{dynamic}{aspect3d}
\begin{Description}\relax
This function sets the apparent ratios of the x, y, and z axes
of the current bounding box.
\end{Description}
\begin{Usage}
\begin{verbatim}
aspect3d(x, y = NULL, z = NULL)
\end{verbatim}
\end{Usage}
\begin{Arguments}
\begin{ldescription}
\item[\code{x}] The ratio for the x axis, or all three ratios, or \code{"iso"} 
\item[\code{y}] The ratio for the y axis 
\item[\code{z}] The ratio for the z axis 
\end{ldescription}
\end{Arguments}
\begin{Details}\relax
If the ratios are all 1, the bounding box will be displayed as a cube approximately filling the display.
Values may be set larger or smaller as desired.  Aspect \code{"iso"} signifies that the
coordinates should all be displayed at the same scale, i.e. the bounding box should not be
rescaled.  (This corresponds to the default display before \code{aspect3d} has been called.)
Partial matches to \code{"iso"} are allowed.

\code{aspect3d} works by modifying \code{par3d("scale")}.
\end{Details}
\begin{Value}
The previous value of the scale is returned invisibly.
\end{Value}
\begin{Author}\relax
Duncan Murdoch
\end{Author}
\begin{SeeAlso}\relax
\code{\LinkA{plot3d}{plot3d}}, \code{\LinkA{par3d}{par3d}}
\end{SeeAlso}
\begin{Examples}
\begin{ExampleCode}
  x <- rnorm(100)
  y <- rnorm(100)*2
  z <- rnorm(100)*3
  
  open3d()
  plot3d(x, y, z)
  aspect3d(1,1,0.5)
  open3d()
  plot3d(x, y, z)
  aspect3d("iso")
\end{ExampleCode}
\end{Examples}


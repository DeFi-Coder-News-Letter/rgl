\HeaderA{ellipse3d}{Make an ellipsoid}{ellipse3d}
\methaliasA{ellipse3d.default}{ellipse3d}{ellipse3d.default}
\methaliasA{ellipse3d.glm}{ellipse3d}{ellipse3d.glm}
\methaliasA{ellipse3d.lm}{ellipse3d}{ellipse3d.lm}
\methaliasA{ellipse3d.nls}{ellipse3d}{ellipse3d.nls}
\keyword{dplot}{ellipse3d}
\begin{Description}\relax
A generic function and several methods
returning an ellipsoid or other outline of a confidence region
for three parameters.
\end{Description}
\begin{Usage}
\begin{verbatim}
ellipse3d(x, ...)
## Default S3 method:
ellipse3d(x, scale = c(1, 1, 1), centre = c(0, 0, 0), level = 0.95, 
            t = sqrt(qchisq(level, 3)), which = 1:3, subdivide = 4, ...)
## S3 method for class 'lm':
ellipse3d(x, which = 1:3, level = 0.95, t = sqrt(3 * qf(level, 
                                                3, x$df.residual)), ...)     
## S3 method for class 'glm':
ellipse3d(x, which = 1:3, level = 0.95, t, dispersion, ...) 
## S3 method for class 'nls':
ellipse3d(x, which = 1:3, level = 0.95, t = sqrt(3 * qf(level, 
                                                3, s$df[2])), ...) 
\end{verbatim}
\end{Usage}
\begin{Arguments}
\begin{ldescription}
\item[\code{x}] An object. In the default method the parameter \code{x} should be a 
square positive definite matrix at least 3x3
in size. It will be treated as the correlation or covariance 
of a multivariate normal distribution.

\item[\code{...}] Additional parameters to pass to the default method or to \code{\LinkA{qmesh3d}{qmesh3d}}.

\item[\code{scale}] If \code{x} is a correlation matrix, then the standard deviations of each
parameter can be given in the scale parameter.  This defaults to \code{c(1, 1, 1)},
so no rescaling will be done.

\item[\code{centre}] The centre of the ellipse will be at this position.

\item[\code{level}] The confidence level of a simulataneous confidence region.  The default is
0.95, for a 95\% region.  This is used to control the size of the ellipsoid.

\item[\code{t}] The size of the ellipse may also be controlled by specifying the value
of a t-statistic on its boundary.  This defaults to the appropriate
value for the confidence region.

\item[\code{which}] This parameter selects which variables from the object will be
plotted.  The default is the first 3.

\item[\code{subdivide}] This controls the number of subdivisions (see \code{\LinkA{subdivision3d}{subdivision3d}})
used in constructing the ellipsoid.  Higher numbers give a smoother shape.

\item[\code{dispersion}] The value of dispersion to use.  If specified, it is treated as fixed,
and chi-square limits for \code{t} are used. If missing, it is 
taken from \code{summary(x)}.

\end{ldescription}
\end{Arguments}
\begin{Value}
A \code{\LinkA{qmesh3d}{qmesh3d}} object representing the ellipsoid.
\end{Value}
\begin{Examples}
\begin{ExampleCode}
# Plot a random sample and an ellipsoid of concentration corresponding to a 95% 
# probability region for a
# trivariate normal distribution with mean 0, unit variances and 
# correlation 0.8.
if (require(MASS)) {
  Sigma <- matrix(c(10,3,0,3,2,0,0,0,1), 3,3)
  Mean <- 1:3
  x <- mvrnorm(1000, Mean, Sigma)
  
  open3d()
  
  plot3d(x, size=3, box=FALSE)
  
  plot3d( ellipse3d(Sigma, centre=Mean), col="green", alpha=0.5, add = TRUE)
}  

# Plot the estimate and joint 90% confidence region for the displacement and cylinder
# count linear coefficients in the mtcars dataset

data(mtcars)
fit <- lm(mpg ~ disp + cyl , mtcars)

open3d()
plot3d(ellipse3d(fit, level = 0.90), col="blue", alpha=0.5, aspect=TRUE)
\end{ExampleCode}
\end{Examples}


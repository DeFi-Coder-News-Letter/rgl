\HeaderA{rgl.primitive}{add primitive set shape}{rgl.primitive}
\aliasA{rgl.lines}{rgl.primitive}{rgl.lines}
\aliasA{rgl.linestrips}{rgl.primitive}{rgl.linestrips}
\aliasA{rgl.points}{rgl.primitive}{rgl.points}
\aliasA{rgl.quads}{rgl.primitive}{rgl.quads}
\aliasA{rgl.triangles}{rgl.primitive}{rgl.triangles}
\keyword{dynamic}{rgl.primitive}
\begin{Description}\relax
Adds a shape node to the current scene
\end{Description}
\begin{Usage}
\begin{verbatim}
rgl.points(x, y = NULL, z = NULL, ... )
rgl.lines(x, y = NULL, z = NULL, ... )
rgl.linestrips(x, y = NULL, z = NULL, ...)
rgl.triangles(x, y = NULL, z = NULL, ... )
rgl.quads(x, y = NULL, z = NULL, ... )
\end{verbatim}
\end{Usage}
\begin{Arguments}
\begin{ldescription}
\item[\code{x, y, z}] coordinates.  Any reasonable way of defining the
coordinates is acceptable.  See the function \code{\LinkA{xyz.coords}{xyz.coords}}
for details.
\item[\code{ ... }] Material properties. See \code{\LinkA{rgl.material}{rgl.material}} for details.
\end{ldescription}
\end{Arguments}
\begin{Details}\relax
Adds a shape node to the scene. The appearance is defined by the material properties.
See \code{\LinkA{rgl.material}{rgl.material}} for details.
\end{Details}
\begin{Value}
Each primitive function returns the integer object ID of the shape that
was added to the scene.  These can be passed to \code{\LinkA{rgl.pop}{rgl.pop}}
to remove the object from the scene.
\end{Value}
\begin{SeeAlso}\relax
\code{\LinkA{rgl.material}{rgl.material}},
\code{\LinkA{rgl.spheres}{rgl.spheres}},
\code{\LinkA{rgl.texts}{rgl.texts}},
\code{\LinkA{rgl.surface}{rgl.surface}},
\code{\LinkA{rgl.sprites}{rgl.sprites}}
\end{SeeAlso}
\begin{Examples}
\begin{ExampleCode}
rgl.open()
rgl.points(rnorm(1000), rnorm(1000), rnorm(1000), color=heat.colors(1000), size=2)
\end{ExampleCode}
\end{Examples}


\HeaderA{rgl.material}{Generic Appearance setup}{rgl.material}
\aliasA{material3d}{rgl.material}{material3d}
\keyword{dynamic}{rgl.material}
\begin{Description}\relax
Set material properties for geometry appearance.
\end{Description}
\begin{Usage}
\begin{verbatim}
rgl.material(
  color        = c("white"),
  alpha        = c(1.0),
  lit          = TRUE, 
  ambient      = "black",
  specular     = "white", 
  emission     = "black", 
  shininess    = 50.0, 
  smooth       = TRUE,
  texture      = NULL, 
  textype      = "rgb", 
  texmipmap    = FALSE, 
  texminfilter = "linear", 
  texmagfilter = "linear",
  texenvmap    = FALSE,
  front        = "fill", 
  back         = "fill",
  size         = 1.0, 
  fog          = TRUE, 
  ...
)
material3d(...)
\end{verbatim}
\end{Usage}
\begin{Arguments}
\begin{ldescription}
\item[\code{color}] vector of R color characters. Represents the diffuse component in case of lighting calculation (lit = TRUE),
otherwise it describes the solid color characteristics.

\item[\code{lit}] logical, specifying if lighting calculation should take place on geometry

\item[\code{ambient, specular, emission, shininess}] properties for lighting calculation. ambient, specular, emission are R color character string values; shininess represents a
numerical.

\item[\code{alpha}] vector of alpha values between 0.0 (fully transparent) .. 1.0 (opaque).

\item[\code{smooth}] logical, specifying whether gourad shading (smooth) or flat shading should be used.

\item[\code{texture}] path to a texture image file. Supported formats: png.

\item[\code{textype}] specifies what is defined with the pixmap
\describe{
\item["alpha"] alpha values
\item["luminance"] luminance
\item["luminance.alpha"] luminance and alpha
\item["rgb"] color
\item["rgba"] color and alpha texture
}

\item[\code{texmipmap}] Logical, specifies if the texture should be mipmapped.

\item[\code{texmagfilter}] specifies the magnification filtering type (sorted by ascending quality):
\describe{
\item["nearest"] texel nearest to the center of the pixel
\item["linear"] weighted linear average of a 2x2 array of texels
}

\item[\code{texminfilter}] specifies the minification filtering type (sorted by ascending quality):
\describe{
\item["nearest"] texel nearest to the center of the pixel
\item["linear"] weighted linear average of a 2x2 array of texels
\item["nearest.mipmap.nearest"] low quality mipmapping
\item["nearest.mipmap.linear"] medium quality mipmapping
\item["linear.mipmap.nearest"] medium quality mipmapping
\item["linear.mipmap.linear"] high quality mipmapping
}

\item[\code{texenvmap}] logical, specifies if auto-generated texture coordinates for environment-mapping 
should be performed on geometry.

\item[\code{front, back}] Determines the polygon mode for the specified side:
\describe{
\item["fill"] filled polygon
\item["line"] wireframed polygon
\item["points"] point polygon
\item["cull"] culled (hidden) polygon
}

\item[\code{size}] numeric, specifying the line and point size.

\item[\code{fog}] logical, specifying if fog effect should be applied on the corresponding shape
\item[\code{...}] Any of the arguments above; see Details below.
\end{ldescription}
\end{Arguments}
\begin{Details}\relax
Only one side at a time can be culled.

\code{material3d} is an alternate interface to the material properties, modelled after
\code{\LinkA{par3d}{par3d}}:  rather than setting defaults for parameters that are not specified, 
they will be left unchanged.  \code{material3d} may also be used to query the material
properties; see the examples below.

The current implementation does not return parameters for textures.

The \code{...} parameter to \code{rgl.material} is ignored.
\end{Details}
\begin{Value}
\code{rgl.material()} is called for the side effect of setting the material properties.
It returns a value invisibly which is not intended for use by the user.

Users should use \code{material3d()} to query material properties.  It returns values similarly
to \code{\LinkA{par3d}{par3d}} as follows:
When setting properties, it returns the previous values in a named list.  A named list is also
returned when more than one value is queried.  When a single value is queried it is returned 
directly.
\end{Value}
\begin{SeeAlso}\relax
\code{\LinkA{rgl.primitive}{rgl.primitive}},
\code{\LinkA{rgl.bbox}{rgl.bbox}},
\code{\LinkA{rgl.bg}{rgl.bg}},
\code{\LinkA{rgl.light}{rgl.light}}
\end{SeeAlso}
\begin{Examples}
\begin{ExampleCode}
save <- material3d("color")
material3d(color="red")
material3d("color")
material3d(color=save)
\end{ExampleCode}
\end{Examples}


\documentclass{article}
\usepackage[ae,hyper]{Rd}
\begin{document}
\HeaderA{points3d}{add primitive set shape}{points3d}
\aliasA{lines3d}{points3d}{lines3d}
\aliasA{quads3d}{points3d}{quads3d}
\aliasA{segments3d}{points3d}{segments3d}
\aliasA{triangles3d}{points3d}{triangles3d}
\keyword{dynamic}{points3d}
\begin{Description}\relax
Adds a shape node to the current scene
\end{Description}
\begin{Usage}
\begin{verbatim}
points3d(x, y = NULL, z = NULL,  ...)
lines3d(x, y = NULL, z = NULL,  ...)
segments3d(x, y = NULL, z = NULL, ...)
triangles3d(x, y = NULL, z = NULL, ...)
quads3d(x, y = NULL, z = NULL, ...)
\end{verbatim}
\end{Usage}
\begin{Arguments}
\begin{ldescription}
\item[\code{x, y, z}] coordinates. Any reasonable way of defining the
coordinates is acceptable.  See the function \code{\LinkA{xyz.coords}{xyz.coords}}
for details.
\item[\code{ ... }] Material properties. See \code{\LinkA{rgl.material}{rgl.material}} for details.
\end{ldescription}
\end{Arguments}
\begin{Details}\relax
The functions \code{points3d}, \code{lines3d}, \code{segments3d},
\code{triangles3d} and \code{quads3d} add points, joined lines, line segments,
filled triangles or quadrilaterals to the plots.  They correspond to the OpenGL types
\code{GL\_POINTS, GL\_LINE\_STRIP, GL\_LINES, GL\_TRIANGLES} and \code{GL\_QUADS} respectively.  

Points are taken in pairs by \code{segments3d}, triplets as the vertices
of the triangles, and quadruplets for the quadrilaterals.  Colours are applied vertex by vertex; 
if different at each end of a line segment, or each vertex of a polygon, the colours
are blended over the extent of the object.  Quadrilaterals must be entirely 
in one plane and convex, or the results are undefined.

These functions call the lower level functions \code{\LinkA{rgl.points}{rgl.points}}, \code{\LinkA{rgl.linestrips}{rgl.linestrips}},
and so on, and are provided for convenience.

The appearance of the new objects are defined by the material properties.
See \code{\LinkA{rgl.material}{rgl.material}} for details.
\end{Details}
\begin{Value}
Each function returns the integer object ID of the shape that
was added to the scene.  These can be passed to \code{\LinkA{rgl.pop}{rgl.pop}}
to remove the object from the scene.
\end{Value}
\begin{Author}\relax
Ming Chen and Duncan Murdoch
\end{Author}
\begin{Examples}
\begin{ExampleCode}
# Show 12 random vertices in various ways. 

M <- matrix(rnorm(36), 3, 12, dimnames=list(c('x','y','z'), 
                                       rep(LETTERS[1:4], 3)))

# Force 4-tuples to be convex in planes so that quads3d works.

for (i in c(1,5,9)) {
    quad <- as.data.frame(M[,i+0:3])
    coeffs <- runif(2,0,3)
    if (mean(coeffs) < 1) coeffs <- coeffs + 1 - mean(coeffs)
    quad$C <- with(quad, coeffs[1]*(B-A) + coeffs[2]*(D-A) + A)
    M[,i+0:3] <- as.matrix(quad)
}

open3d()

# Rows of M are x, y, z coords; transpose to plot

M <- t(M)
shift <- matrix(c(-3,3,0), 12, 3, byrow=TRUE)

points3d(M, size=2)
lines3d(M + shift)
segments3d(M + 2*shift)
triangles3d(M + 3*shift, col='red')
quads3d(M + 4*shift, col='green')  
text3d(M + 5*shift, texts=1:12)

# Add labels

shift <- outer(0:5, shift[1,])
shift[,1] <- shift[,1] + 3
text3d(shift, 
       texts = c('points3d','lines3d','segments3d',
         'triangles3d', 'quads3d','text3d'),
       adj = 0)
 rgl.bringtotop()
\end{ExampleCode}
\end{Examples}

\HeaderA{aspect3d}{Set the aspect ratios of the current plot}{aspect3d}
\keyword{dynamic}{aspect3d}
\begin{Description}\relax
This function sets the apparent ratios of the x, y, and z axes
of the current bounding box.
\end{Description}
\begin{Usage}
\begin{verbatim}
aspect3d(x, y = NULL, z = NULL)
\end{verbatim}
\end{Usage}
\begin{Arguments}
\begin{ldescription}
\item[\code{x}] The ratio for the x axis, or all three ratios, or \code{"iso"} 
\item[\code{y}] The ratio for the y axis 
\item[\code{z}] The ratio for the z axis 
\end{ldescription}
\end{Arguments}
\begin{Details}\relax
If the ratios are all 1, the bounding box will be displayed as a cube approximately filling the display.
Values may be set larger or smaller as desired.  Aspect \code{"iso"} signifies that the
coordinates should all be displayed at the same scale, i.e. the bounding box should not be
rescaled.  (This corresponds to the default display before \code{aspect3d} has been called.)
Partial matches to \code{"iso"} are allowed.

\code{aspect3d} works by modifying \code{par3d("scale")}.
\end{Details}
\begin{Value}
The previous value of the scale is returned invisibly.
\end{Value}
\begin{Author}\relax
Duncan Murdoch
\end{Author}
\begin{SeeAlso}\relax
\code{\LinkA{plot3d}{plot3d}}, \code{\LinkA{par3d}{par3d}}
\end{SeeAlso}
\begin{Examples}
\begin{ExampleCode}
  x <- rnorm(100)
  y <- rnorm(100)*2
  z <- rnorm(100)*3
  
  open3d()
  plot3d(x, y, z)
  aspect3d(1,1,0.5)
  open3d()
  plot3d(x, y, z)
  aspect3d("iso")
\end{ExampleCode}
\end{Examples}

\HeaderA{axes3d}{Draw boxes, axes and other text outside the data}{axes3d}
\aliasA{axis3d}{axes3d}{axis3d}
\aliasA{box3d}{axes3d}{box3d}
\aliasA{mtext3d}{axes3d}{mtext3d}
\aliasA{title3d}{axes3d}{title3d}
\keyword{dynamic}{axes3d}
\begin{Description}\relax
These functions draw axes, boxes and text outside the range of the data.
\code{axes3d}, \code{box3d} and \code{title3d} are the higher level functions; 
normally the others need not be called directly by users.
\end{Description}
\begin{Usage}
\begin{verbatim}
axes3d(edges = "bbox", labels = TRUE, tick = TRUE, nticks = 5, ...)
box3d(...) 
title3d(main = NULL, sub = NULL, xlab = NULL, ylab = NULL, 
    zlab = NULL, line = NA, ...) 
axis3d(edge, at = NULL, labels = TRUE, tick = TRUE, line = 0, 
    pos = NULL, nticks = 5, ...) 
mtext3d(text, edge, line = 0, at = NULL, pos = NA, ...) 
\end{verbatim}
\end{Usage}
\begin{Arguments}
\begin{ldescription}
\item[\code{edges}] a code to describe which edge(s) of the box to use; see Details below 
\item[\code{labels}] whether to label the axes, or (for \code{axis3d}) the
labels to use
\item[\code{tick}] whether to use tick marks 
\item[\code{nticks}] suggested number of ticks 
\item[\code{main}] the main title for the plot 
\item[\code{sub}] the subtitle for the plot 
\item[\code{xlab, ylab, zlab}] the axis labels for the plot 
\item[\code{line}] the ``line'' of the plot margin to draw the label on 
\item[\code{edge, pos}] the position at which to draw the axis or text 
\item[\code{text}] the text to draw 
\item[\code{at}] the value of a coordinate at which to draw the axis 
\item[\code{...}] additional parameters which are passed to \code{\LinkA{bbox3d}{bbox3d}} or \code{\LinkA{material3d}{material3d}} 
\end{ldescription}
\end{Arguments}
\begin{Details}\relax
The rectangular prism holding the 3D plot has 12 edges.  They are identified
using 3 character strings.  The first character (`x', `y', or `z') selects 
the direction of the axis.  The next two characters are each `-' or `+',
selecting the lower or upper end of one of the other coordinates.  If only
one or two characters are given, the remaining characters default to `-'.  
For example \code{edge = 'x+'} draws an x-axis at the high level of y and the
low level of z.

By default, \code{axes3d} uses the \code{\LinkA{bbox3d}{bbox3d}} function to draw the axes.  
The labels will move so that they do not obscure the data.  Alternatively,
a vector of arguments as described above may be used, in which case
fixed axes are drawn using \code{axis3d}.

If \code{pos} is a numeric vector of length 3, \code{edge} determines
the direction of the axis and the tick marks, and the values of the
other two coordinates in \code{pos} determine the position.  See the
examples.
\end{Details}
\begin{Value}
These functions are called for their side effects.  They return the object IDs of
objects added to the scene.
\end{Value}
\begin{Author}\relax
Duncan Murdoch
\end{Author}
\begin{SeeAlso}\relax
\code{\LinkA{axis}{axis}}, \code{\LinkA{box}{box}},
\code{title}, \code{mtext}, \LinkA{bbox3d}{bbox3d}
\end{SeeAlso}
\begin{Examples}
\begin{ExampleCode}
  open3d()
  points3d(rnorm(10),rnorm(10),rnorm(10), size=3)

  # First add standard axes
  axes3d()  

  # and one in the middle (the NA will be ignored, a number would 
  # do as well)
  axis3d('x',pos=c(NA, 0, 0))

  # add titles
  title3d('main','sub','xlab','ylab','zlab')

  rgl.bringtotop()
  
  open3d()
  points3d(rnorm(10),rnorm(10),rnorm(10), size=3)
  
  # Use fixed axes
  
  axes3d(c('x','y','z'))
         
  # Put 4 x-axes on the plot
  axes3d(c('x--','x-+','x+-','x++'))         
  
  axis3d('x',pos=c(NA, 0, 0))     
  title3d('main','sub','xlab','ylab','zlab')
\end{ExampleCode}
\end{Examples}

\HeaderA{rgl.bbox}{Set up Bounding Box decoration}{rgl.bbox}
\aliasA{bbox3d}{rgl.bbox}{bbox3d}
\keyword{dynamic}{rgl.bbox}
\begin{Description}\relax
Set up the bounding box decoration.
\end{Description}
\begin{Usage}
\begin{verbatim}
rgl.bbox( 
  xat=NULL, xlab=NULL, xunit=0, xlen=5, 
  yat=NULL, ylab=NULL, yunit=0, ylen=5,
  zat=NULL, zlab=NULL, zunit=0, zlen=5,
  marklen=15.0, marklen.rel=TRUE, expand=1, ...)
bbox3d(xat, yat, zat, expand=1.03, nticks=5, ...)  
\end{verbatim}
\end{Usage}
\begin{Arguments}
\begin{ldescription}
\item[\code{xat,yat,zat}] vector specifying the tickmark positions
\item[\code{xlab,ylab,zlab}] character vector specifying the tickmark labeling
\item[\code{xunit,yunit,zunit}] value specifying the tick mark base for uniform tick mark layout
\item[\code{xlen,ylen,zlen}] value specifying the number of tickmarks
\item[\code{marklen}] value specifying the length of the tickmarks
\item[\code{marklen.rel}] logical, if TRUE tick mark length is calculated using 1/\code{marklen} * axis length, otherwise tick mark length is \code{marklen} in coordinate space
\item[\code{expand}] value specifying how much to expand the bounding box around the data
\item[\code{nticks}] suggested number of ticks to use on axes
\item[\code{ ... }] Material properties (or other \code{rgl.bbox} parameters
in the case of \code{bbox3d}). See \code{\LinkA{rgl.material}{rgl.material}} for details.
\end{ldescription}
\end{Arguments}
\begin{Details}\relax
Three different types of tick mark layouts are possible.
If \code{at} is not \code{NULL}, the ticks are set up at custom positions.
If \code{unit} is not zero, it defines the tick mark base.
If \code{length} is not zero, it specifies the number of ticks that are automatically specified.
The first colour specifies the bounding box, while the second one specifies the tick mark and font colour.

\code{bbox3d} defaults to \code{\LinkA{pretty}{pretty}} locations for the axis labels and a slightly larger
box, whereas \code{rgl.bbox} covers the exact range.

\code{\LinkA{axes3d}{axes3d}} offers more flexibility in the specification of the axes, but 
they are static, unlike those drawn by \code{\LinkA{rgl.bbox}{rgl.bbox}} and \code{\LinkA{bbox3d}{bbox3d}}.
\end{Details}
\begin{Value}
This function is called for the side effect of setting the bounding box decoration.
A shape ID is returned to allow \code{\LinkA{rgl.pop}{rgl.pop}} to delete it.
\end{Value}
\begin{SeeAlso}\relax
\code{\LinkA{rgl.material}{rgl.material}}, \code{\LinkA{axes3d}{axes3d}}
\end{SeeAlso}
\begin{Examples}
\begin{ExampleCode}
  rgl.open()
  rgl.points(rnorm(100), rnorm(100), rnorm(100))
  rgl.bbox(color=c("#333377","white"), emission="#333377", 
           specular="#3333FF", shininess=5, alpha=0.8 )
  
  open3d()
  points3d(rnorm(100), rnorm(100), rnorm(100))
  bbox3d(color=c("#333377","white"), emission="#333377", 
         specular="#3333FF", shininess=5, alpha=0.8)
\end{ExampleCode}
\end{Examples}

\HeaderA{bg}{Set up Background}{bg}
\aliasA{bg3d}{bg}{bg3d}
\aliasA{rgl.bg}{bg}{rgl.bg}
\keyword{dynamic}{bg}
\begin{Description}\relax
Setup the background environment of the scene.
\end{Description}
\begin{Usage}
\begin{verbatim}
bg3d(...) 
rgl.bg( sphere = FALSE, fogtype = "none", color=c("black","white"), 
        back="lines", ...)

\end{verbatim}
\end{Usage}
\begin{Arguments}
\begin{ldescription}
\item[\code{fogtype}] fog type:
\describe{
\item["none"] no fog
\item["linear"] linear fog function
\item["exp"] exponential fog function
\item["exp2"] squared exponential fog function
}

\item[\code{sphere}] logical, if true, an environmental sphere geometry is used for the background decoration.

\item[\code{color}] Primary color is used for background clearing and as fog color.
Secondary color is used for background sphere geometry. See \code{\LinkA{rgl.material}{rgl.material}} for details.

\item[\code{back}] Specifies the fill style of the sphere geometry. See \code{\LinkA{rgl.material}{rgl.material}} for details.

\item[\code{ ... }] Material properties. See \code{\LinkA{rgl.material}{rgl.material}} for details.
\end{ldescription}
\end{Arguments}
\begin{Details}\relax
If sphere is set to TRUE, an environmental sphere enclosing the whole scene is drawn.
\end{Details}
\begin{SeeAlso}\relax
\code{\LinkA{rgl.material}{rgl.material}}
\end{SeeAlso}
\begin{Examples}
\begin{ExampleCode}
  rgl.open()
  
  # a simple white background
  
  bg3d("white")

  # the holo-globe (inspired by star trek):

  rgl.bg(sphere=TRUE, color=c("black","green"), lit=FALSE, back="lines" )

  # an environmental sphere with a nice texture.

  rgl.bg(sphere=TRUE, texture=system.file("textures/sunsleep.png", package="rgl"), 
         back="filled" )
\end{ExampleCode}
\end{Examples}

\HeaderA{ellipse3d}{Make an ellipsoid}{ellipse3d}
\methaliasA{ellipse3d.default}{ellipse3d}{ellipse3d.default}
\methaliasA{ellipse3d.glm}{ellipse3d}{ellipse3d.glm}
\methaliasA{ellipse3d.lm}{ellipse3d}{ellipse3d.lm}
\methaliasA{ellipse3d.nls}{ellipse3d}{ellipse3d.nls}
\keyword{dplot}{ellipse3d}
\begin{Description}\relax
A generic function and several methods
returning an ellipsoid or other outline of a confidence region
for three parameters.
\end{Description}
\begin{Usage}
\begin{verbatim}
ellipse3d(x, ...)
## Default S3 method:
ellipse3d(x, scale = c(1, 1, 1), centre = c(0, 0, 0), level = 0.95, 
            t = sqrt(qchisq(level, 3)), which = 1:3, subdivide = 4, ...)
## S3 method for class 'lm':
ellipse3d(x, which = 1:3, level = 0.95, t = sqrt(3 * qf(level, 
                                                3, x$df.residual)), ...)     
## S3 method for class 'glm':
ellipse3d(x, which = 1:3, level = 0.95, t, dispersion, ...) 
## S3 method for class 'nls':
ellipse3d(x, which = 1:3, level = 0.95, t = sqrt(3 * qf(level, 
                                                3, s$df[2])), ...) 
\end{verbatim}
\end{Usage}
\begin{Arguments}
\begin{ldescription}
\item[\code{x}] An object. In the default method the parameter \code{x} should be a 
square positive definite matrix at least 3x3
in size. It will be treated as the correlation or covariance 
of a multivariate normal distribution.

\item[\code{...}] Additional parameters to pass to the default method or to \code{\LinkA{qmesh3d}{qmesh3d}}.

\item[\code{scale}] If \code{x} is a correlation matrix, then the standard deviations of each
parameter can be given in the scale parameter.  This defaults to \code{c(1, 1, 1)},
so no rescaling will be done.

\item[\code{centre}] The centre of the ellipse will be at this position.

\item[\code{level}] The confidence level of a simulataneous confidence region.  The default is
0.95, for a 95\% region.  This is used to control the size of the ellipsoid.

\item[\code{t}] The size of the ellipse may also be controlled by specifying the value
of a t-statistic on its boundary.  This defaults to the appropriate
value for the confidence region.

\item[\code{which}] This parameter selects which variables from the object will be
plotted.  The default is the first 3.

\item[\code{subdivide}] This controls the number of subdivisions (see \code{\LinkA{subdivision3d}{subdivision3d}})
used in constructing the ellipsoid.  Higher numbers give a smoother shape.

\item[\code{dispersion}] The value of dispersion to use.  If specified, it is treated as fixed,
and chi-square limits for \code{t} are used. If missing, it is 
taken from \code{summary(x)}.

\end{ldescription}
\end{Arguments}
\begin{Value}
A \code{\LinkA{qmesh3d}{qmesh3d}} object representing the ellipsoid.
\end{Value}
\begin{Examples}
\begin{ExampleCode}
# Plot a random sample and an ellipsoid of concentration corresponding to a 95% 
# probability region for a
# trivariate normal distribution with mean 0, unit variances and 
# correlation 0.8.
if (require(MASS)) {
  Sigma <- matrix(c(10,3,0,3,2,0,0,0,1), 3,3)
  Mean <- 1:3
  x <- mvrnorm(1000, Mean, Sigma)
  
  open3d()
  
  plot3d(x, size=3, box=FALSE)
  
  plot3d( ellipse3d(Sigma, centre=Mean), col="green", alpha=0.5, add = TRUE)
}  

# Plot the estimate and joint 90% confidence region for the displacement and cylinder
# count linear coefficients in the mtcars dataset

data(mtcars)
fit <- lm(mpg ~ disp + cyl , mtcars)

open3d()
plot3d(ellipse3d(fit, level = 0.90), col="blue", alpha=0.5, aspect=TRUE)
\end{ExampleCode}
\end{Examples}

\HeaderA{grid3d}{Add a grid to a 3D plot}{grid3d}
\keyword{dynamic}{grid3d}
\begin{Description}\relax
This function adds a reference grid to an RGL plot.
\end{Description}
\begin{Usage}
\begin{verbatim}
grid3d(side, at = NULL, col = "gray", lwd = 1, lty = 1, n = 5)
\end{verbatim}
\end{Usage}
\begin{Arguments}
\begin{ldescription}
\item[\code{side}] Where to put the grid; see the Details section. 
\item[\code{at}] How to draw the grid; see the Details section. 
\item[\code{col}] The color of the grid lines. 
\item[\code{lwd}] The line width of the grid lines. 
\item[\code{lty}] The line type of the grid lines. 
\item[\code{n}] Suggested number of grid lines; see the Details section. 
\end{ldescription}
\end{Arguments}
\begin{Details}\relax
This function is similar to \code{\LinkA{grid}{grid}} in classic graphics,
except that it draws a 3D grid in the plot.

The grid is drawn in a plane perpendicular to the coordinate axes. The
first letter of the \code{side} argument specifies the direction of
the plane: \code{"x"}, \code{"y"} or \code{"z"} (or uppercase
versions) to specify the coordinate which is constant on the plane.

If \code{at = NULL} (the default), the grid is drawn at the limit of 
the box around the data.  If the second letter of the \code{side} argument
is \code{"-"} or is not present, it is the lower limit; if \code{"+"}
then at the upper limit.  The grid lines are drawn at values
chosen by \code{\LinkA{pretty}{pretty}} with \code{n} suggested locations.
The default locations should match those chosen by \code{\LinkA{axis3d}{axis3d}}
with \code{nticks = n}.

If \code{at} is a numeric vector, the grid lines are drawn at those values.

If \code{at} is a list, then the \code{"x"} component is used to
specify the x location, the \code{"y"} component specifies the y location, and
the \code{"z"} component specifies the z location.  Missing components
are handled using the default as for \code{at = NULL}.

Multiple grids may be drawn by specifying multiple values for \code{side}
or for the component of \code{at} that specifies the grid location.
\end{Details}
\begin{Value}
A vector or matrix of object ids is returned invisibly.
\end{Value}
\begin{Author}\relax
Ben Bolker and Duncan Murdoch
\end{Author}
\begin{SeeAlso}\relax
\code{\LinkA{axis3d}{axis3d}}
\end{SeeAlso}
\begin{Examples}
\begin{ExampleCode}
x <- 1:10
y <- 1:10
z <- matrix(outer(x-5,y-5) + rnorm(100), 10, 10)
open3d()
persp3d(x, y, z, col="red", alpha=0.7, aspect=c(1,1,0.5))
grid3d(c("x", "y+", "z"))
\end{ExampleCode}
\end{Examples}

\HeaderA{light}{add light source}{light}
\aliasA{light3d}{light}{light3d}
\aliasA{rgl.light}{light}{rgl.light}
\keyword{dynamic}{light}
\begin{Description}\relax
add a light source to the scene.
\end{Description}
\begin{Usage}
\begin{verbatim}
light3d(theta = 0, phi = 15, ...)
rgl.light( theta = 0, phi = 0, viewpoint.rel = TRUE, ambient = "#FFFFFF", 
           diffuse = "#FFFFFF", specular = "#FFFFFF")
\end{verbatim}
\end{Usage}
\begin{Arguments}
\begin{ldescription}
\item[\code{theta, phi}] polar coordinates
\item[\code{viewpoint.rel}] logical, if TRUE light is a viewpoint light that is positioned relative to the current viewpoint
\item[\code{ambient, diffuse, specular }] 
\item[\code{...}] generic arguments passed through to RGL-specific (or other) functions
\end{ldescription}
\end{Arguments}
\begin{Details}\relax
Up to 8 light sources are supported. They are positioned either in world space
or relative to the camera using polar coordinates. Light sources are directional.
\end{Details}
\begin{Value}
This function is called for the side effect of adding a light.  A light ID is
returned to allow \code{\LinkA{rgl.pop}{rgl.pop}} to remove it.
\end{Value}
\begin{SeeAlso}\relax
\code{\LinkA{rgl.clear}{rgl.clear}}
\code{\LinkA{rgl.pop}{rgl.pop}}
\end{SeeAlso}

\HeaderA{rgl.material}{Generic Appearance setup}{rgl.material}
\aliasA{material3d}{rgl.material}{material3d}
\keyword{dynamic}{rgl.material}
\begin{Description}\relax
Set material properties for geometry appearance.
\end{Description}
\begin{Usage}
\begin{verbatim}
rgl.material(
  color        = c("white"),
  alpha        = c(1.0),
  lit          = TRUE, 
  ambient      = "black",
  specular     = "white", 
  emission     = "black", 
  shininess    = 50.0, 
  smooth       = TRUE,
  texture      = NULL, 
  textype      = "rgb", 
  texmipmap    = FALSE, 
  texminfilter = "linear", 
  texmagfilter = "linear",
  texenvmap    = FALSE,
  front        = "fill", 
  back         = "fill",
  size         = 1.0, 
  fog          = TRUE, 
  ...
)
material3d(...)
\end{verbatim}
\end{Usage}
\begin{Arguments}
\begin{ldescription}
\item[\code{color}] vector of R color characters. Represents the diffuse component in case of lighting calculation (lit = TRUE),
otherwise it describes the solid color characteristics.

\item[\code{lit}] logical, specifying if lighting calculation should take place on geometry

\item[\code{ambient, specular, emission, shininess}] properties for lighting calculation. ambient, specular, emission are R color character string values; shininess represents a
numerical.

\item[\code{alpha}] vector of alpha values between 0.0 (fully transparent) .. 1.0 (opaque).

\item[\code{smooth}] logical, specifying whether gourad shading (smooth) or flat shading should be used.

\item[\code{texture}] path to a texture image file. Supported formats: png.

\item[\code{textype}] specifies what is defined with the pixmap
\describe{
\item["alpha"] alpha values
\item["luminance"] luminance
\item["luminance.alpha"] luminance and alpha
\item["rgb"] color
\item["rgba"] color and alpha texture
}

\item[\code{texmipmap}] Logical, specifies if the texture should be mipmapped.

\item[\code{texmagfilter}] specifies the magnification filtering type (sorted by ascending quality):
\describe{
\item["nearest"] texel nearest to the center of the pixel
\item["linear"] weighted linear average of a 2x2 array of texels
}

\item[\code{texminfilter}] specifies the minification filtering type (sorted by ascending quality):
\describe{
\item["nearest"] texel nearest to the center of the pixel
\item["linear"] weighted linear average of a 2x2 array of texels
\item["nearest.mipmap.nearest"] low quality mipmapping
\item["nearest.mipmap.linear"] medium quality mipmapping
\item["linear.mipmap.nearest"] medium quality mipmapping
\item["linear.mipmap.linear"] high quality mipmapping
}

\item[\code{texenvmap}] logical, specifies if auto-generated texture coordinates for environment-mapping 
should be performed on geometry.

\item[\code{front, back}] Determines the polygon mode for the specified side:
\describe{
\item["fill"] filled polygon
\item["line"] wireframed polygon
\item["points"] point polygon
\item["cull"] culled (hidden) polygon
}

\item[\code{size}] numeric, specifying the line and point size.

\item[\code{fog}] logical, specifying if fog effect should be applied on the corresponding shape
\item[\code{...}] Any of the arguments above; see Details below.
\end{ldescription}
\end{Arguments}
\begin{Details}\relax
Only one side at a time can be culled.

\code{material3d} is an alternate interface to the material properties, modelled after
\code{\LinkA{par3d}{par3d}}:  rather than setting defaults for parameters that are not specified, 
they will be left unchanged.  \code{material3d} may also be used to query the material
properties; see the examples below.

The current implementation does not return parameters for textures.

The \code{...} parameter to \code{rgl.material} is ignored.
\end{Details}
\begin{Value}
\code{rgl.material()} is called for the side effect of setting the material properties.
It returns a value invisibly which is not intended for use by the user.

Users should use \code{material3d()} to query material properties.  It returns values similarly
to \code{\LinkA{par3d}{par3d}} as follows:
When setting properties, it returns the previous values in a named list.  A named list is also
returned when more than one value is queried.  When a single value is queried it is returned 
directly.
\end{Value}
\begin{SeeAlso}\relax
\code{\LinkA{rgl.primitive}{rgl.primitive}},
\code{\LinkA{rgl.bbox}{rgl.bbox}},
\code{\LinkA{rgl.bg}{rgl.bg}},
\code{\LinkA{rgl.light}{rgl.light}}
\end{SeeAlso}
\begin{Examples}
\begin{ExampleCode}
save <- material3d("color")
material3d(color="red")
material3d("color")
material3d(color=save)
\end{ExampleCode}
\end{Examples}

\HeaderA{matrices}{Work with homogeneous coordinates}{matrices}
\aliasA{asEuclidean}{matrices}{asEuclidean}
\aliasA{asHomogeneous}{matrices}{asHomogeneous}
\aliasA{identityMatrix}{matrices}{identityMatrix}
\aliasA{rotate3d}{matrices}{rotate3d}
\aliasA{rotationMatrix}{matrices}{rotationMatrix}
\aliasA{scale3d}{matrices}{scale3d}
\aliasA{scaleMatrix}{matrices}{scaleMatrix}
\aliasA{transform3d}{matrices}{transform3d}
\aliasA{translate3d}{matrices}{translate3d}
\aliasA{translationMatrix}{matrices}{translationMatrix}
\keyword{dynamic}{matrices}
\begin{Description}\relax
These functions construct 4x4 matrices for transformations
in the homogeneous coordinate system used by OpenGL, and translate
vectors between homogeneous and Euclidean coordinates.
\end{Description}
\begin{Usage}
\begin{verbatim}
identityMatrix()
scaleMatrix(x, y, z)
translationMatrix(x, y, z)
rotationMatrix(angle, x, y, z, matrix)
asHomogeneous(x)
asEuclidean(x)

scale3d(obj, x, y, z, ...)
translate3d(obj, x, y, z, ...)
rotate3d(obj, angle, x, y, z, matrix, ...)

transform3d(obj, matrix, ...)

\end{verbatim}
\end{Usage}
\begin{Arguments}
\begin{ldescription}
\item[\code{x, y, z, angle, matrix}] See details
\item[\code{obj}] An object to be transformed
\item[\code{...}] Additional parameters to be passed to methods
\end{ldescription}
\end{Arguments}
\begin{Details}\relax
OpenGL uses homogeneous coordinates to handle perspective and affine
transformations.  The homogeneous point \code{(x, y, z, w)} corresponds
to the Euclidean point \code{(x/w, y/w, z/w)}.  The matrices produced by
the functions \code{scaleMatrix}, \code{translationMatrix}, and \code{rotationMatrix}
are to be left-multiplied by a row vector 
of homogeneous coordinates; alternatively, the transpose of the result
can be right-multiplied by a column vector.  The generic functions
\code{scale3d}, \code{translate3d} and \code{rotate3d} apply these transformations
to the \code{obj} argument.  The \code{transform3d} function is a synonym 
for \code{rotate3d(obj, matrix=matrix)}.

By default, it is assumed that \code{obj} is a row vector
(or a matrix of row vectors) which will be multiplied on the right by
the corresponding matrix, but users may write methods for these generics
which operate differently.  Methods are supplied for \code{\LinkA{qmesh3d}{qmesh3d}}
objects.

To compose transformations, use matrix multiplication.  The effect is
to apply the matrix on the left first, followed by the one on the right.

\code{identityMatrix} returns an identity matrix.

\code{scaleMatrix} scales each coordinate by the given factor.  In Euclidean
coordinates, \code{(u, v, w)} is transformed to \code{(x*u, y*v, z*w)}.

\code{translationMatrix} translates each coordinate by the given translation, i.e.
\code{(u,v,w)} is transformed to \code{(u+x,v+y,w+z)}.

\code{rotationMatrix} can be called in three ways.  With 
arguments \code{angle, x, y, z} it represents a rotation 
of \code{angle} radians about the axis 
\code{x, y, z}.  If a 3x3 rotation matrix can be given;
it will be converted into the corresponding matrix in 4x4 homogeneous
coordinates.  Finally, if a 4x4 matrix is given, it will be returned unchanged.
(The latter behaviour is used to allow \code{transform3d} to act like a 
generic function, even though it is not.)

Use \code{asHomogeneous(x)} to convert the Euclidean vector \code{x} to
homogeneous coordinates, and \code{asEuclidean(x)} for the reverse transformation.
\end{Details}
\begin{Value}
\code{identityMatrix},
\code{scaleMatrix}, \code{translationMatrix}, and \code{rotationMatrix} produce
a 4x4 matrix representing the requested transformation
in homogeneous coordinates.  

\code{scale3d}, \code{translate3d} and \code{rotate3d} transform the object
and produce a new object of the same class.
\end{Value}
\begin{Author}\relax
Duncan Murdoch
\end{Author}
\begin{SeeAlso}\relax
\code{\LinkA{par3d}{par3d}} for a description of how rgl uses matrices in
rendering.
\end{SeeAlso}
\begin{Examples}
\begin{ExampleCode}
# A 90 degree rotation about the x axis:

rotationMatrix(pi/2, 1, 0, 0)

# Find what happens when you rotate (2,0,0) by 45 degrees about the y axis:

x <- asHomogeneous(c(2,0,0))
y <- x 
asEuclidean(y)

# or more simply...

rotate3d(c(2,0,0), pi/4, 0, 1, 0)

\end{ExampleCode}
\end{Examples}

\HeaderA{par3d}{Set or Query RGL Parameters}{par3d}
\aliasA{open3d}{par3d}{open3d}
\aliasA{r3dDefaults}{par3d}{r3dDefaults}
\keyword{dynamic}{par3d}
\begin{Description}\relax
\code{par3d} can be used to set or query graphical parameters in rgl.
Parameters can be set by specifying them as arguments to \code{par3d} in
\code{tag = value} form, or by passing them as a list of tagged
values.
\end{Description}
\begin{Usage}
\begin{verbatim}
par3d(..., no.readonly = FALSE)

open3d(..., params=get("r3dDefaults", envir=.GlobalEnv))

\end{verbatim}
\end{Usage}
\begin{Arguments}
\begin{ldescription}
\item[\code{...}] arguments in \code{tag = value} form, or a list of tagged
values.  The tags must come from the graphical parameters described
below.
\item[\code{no.readonly}] logical; if \code{TRUE} and there are no other
arguments, only those parameters which can be set by a
subsequent \code{par3d()} call are returned.
\item[\code{params}] a list of graphical parameters
\end{ldescription}
\end{Arguments}
\begin{Details}\relax
Parameters are queried by giving one or more character vectors to
\code{par3d}.

\code{par3d()} (no arguments) or \code{par3d(no.readonly=TRUE)} is used to
get \emph{all} the graphical parameters (as a named list).  

\emph{\bold{R.O.}} indicates \emph{\bold{read-only arguments}}: These
may only be used in queries, i.e., they do \emph{not} set anything.

\code{open3d} opens a new rgl device, and sets the parameters as
requested.  The \code{r3dDefaults} list will be used as default
values for parameters.  As installed this sets the point of view to
'world coordinates' (i.e. x running from left to right, y from front
to back, z from bottom to top), the \code{mouseMode} to
\code{(zAxis, zoom, fov)}, and the field of view to 30 degrees.
Users may create their own variable of that name in the global
environment and it will override the installed one.  If there
is a \code{bg} element in the list or the arguments, it should be
a list of arguments to pass to the \code{\LinkA{bg3d}{bg3d}} function to 
set the background.  A \code{material} element should be a list
of arguments to pass to \code{\LinkA{material3d}{material3d}} to set the material.
\end{Details}
\begin{Value}
When parameters are set, their former values are returned in an
invisible named list.  Such a list can be passed as an argument to
\code{par3d} to restore the parameter values.
Use \code{par3d(no.readonly = TRUE)} for the full list of parameters
that can be restored.

When just one parameter is queried, its value is returned directly.
When two or more parameters are queried, the result is a list of
values, with the list names giving the parameters.

Note the inconsistency: setting one parameter returns a list, but
querying one parameter returns an object.
\end{Value}
\begin{Section}{Parameters}
\describe{
\item[\code{FOV}] real.  The field of view, from 1 to 179 degrees.  This controls
the degree of parallax in the perspective view.
Isometric perspective (which would
correspond to \code{FOV=0}) is not currently possible, but one can
approximate it by specifying \code{par3d(FOV=1)}.
\item[\code{ignoreExtent}] logical.  Set to \code{TRUE} so that subsequently plotted objects
will be ignored in calculating the bounding box of the scene.
\item[\code{modelMatrix}] \emph{\bold{R.O.}}; a 4 by 4 matrix describing the position
of the user data.
\item[\code{mouseMode}] character.  A vector of 3 strings describing
what the 3 mouse buttons do. Partial matching is used.  Possible values for \code{mouseMode} are 
\describe{
\item[\code{"none"}] No action for this button.
\item[\code{"trackball"}] Mouse acts as a virtual trackball, rotating the scene.
\item[\code{"xAxis"}] Similar to \code{"trackball"}, but restricted to X axis rotation.
\item[\code{"yAxis"}] Y axis rotation.
\item[\code{"zAxis"}] Z axis rotation.
\item[\code{"polar"}] Mouse rotates the scene by moving in polar coordinates.
\item[\code{"selecting"}] Mouse is used for selection.  This is not normally set by the
user, but is used internally by the \code{\LinkA{select3d}{select3d}} function.
\item[\code{"zoom"}] Mouse is used to zoom the display.
\item[\code{"fov"}] Mouse changes the field of view of the display.
}

\item[\code{projMatrix}] \emph{\bold{R.O.}}; a 4 by 4 matrix describing the current
projection of the scene.
\item[\code{scale}] real.  A vector of 3 values indicating the amount by which to
rescale each axis before display.  Set by \code{\LinkA{aspect3d}{aspect3d}}.
\item[\code{skipRedraw}] whether to update the display.  Set to \code{FALSE} to suspend
updating while making multiple changes to the scene.  See \code{demo(hist3d)} for an example.
\item[\code{userMatrix}] a 4 by 4 matrix describing user actions to
display the scene.
\item[\code{viewport}] \emph{\bold{R.O.}}; real.  A vector giving the dimensions of the
window in pixels.
\item[\code{zoom}] real. A positive value indicating the current
magnification of the scene.
\item[\code{bbox}] \emph{\bold{R.O.}}; real. A vector of six values indicating the current values of the
bounding box of the scene (xmin, xmax, ymin, ymax, zmin, zmax)
}
\end{Section}
\begin{Section}{note}
The \code{"xAxis"}, \code{"yAxis"} and \code{"zAxis"} mouse modes rotate relative to the coordinate
system of the data, regardless of the current orientation of the scene.
\end{Section}
\begin{Section}{Rendering}
The parameters returned by \code{par3d} are sufficient to determine where rgl would render
a point on the screen.  Given a column vector \code{(x,y,z)}, it performs the equivalent of 
the following operations:
\Enumerate{
\item It converts the point to homogeneous coordinates by appending \code{w=1}, giving
the vector \code{v = (x,y,z,1)}.
\item It obtains the model matrix \code{M = par3d("modelMatrix")} and multiplies
the point by this giving \code{u = M \%*\% v}.  Using this location and information on the
normals (which have been similarly transformed), it performs lighting calculations.  (The 
\code{par3d("userMatrix")} is used in the construction of the model matrix, along with
other transformations to center the object being displayed.)
\item It obtains the projection matrix \code{P = par3d("projMatrix")} and
multiplies the point by it giving \code{P \%*\% u = (x2, y2, z2, w2)}.
\item It converts back to Euclidean coordinates by dividing the first 3 coordinates by \code{w2}.
\item The new value \code{z2/w2} represents the depth into the scene of the point.  Depending
on what has already been plotted, this depth might be obscured, in which case nothing more is plotted.
\item If the point is not culled due to depth, the \code{x2} and \code{y2} values are used
to determine the point in the image.  The \code{par3d("viewport")} values are used to translate
from the range \code{(-1, 1)} to pixel locations, and the point is plotted.
}
See ?\LinkA{matrices}{matrices} for more information on homogeneous and Euclidean coordinates.
\end{Section}
\begin{References}\relax
OpenGL Architecture Review Board (1997).  OpenGL Programming Guide.  Addison-Wesley.
\end{References}
\begin{SeeAlso}\relax
\code{\LinkA{rgl.viewpoint}{rgl.viewpoint}} to set \code{FOV} and \code{zoom}.
\end{SeeAlso}
\begin{Examples}
\begin{ExampleCode}
    r3dDefaults
    open3d()
    shade3d(cube3d(color=rep(rainbow(6),rep(4,6))))
    save <- par3d(userMatrix = rotationMatrix(90*pi/180, 1,0,0))
    save
    par3d("userMatrix")    
    par3d(save)
    par3d("userMatrix")
\end{ExampleCode}
\end{Examples}

\HeaderA{persp3d}{Surface plots}{persp3d}
\methaliasA{persp3d.default}{persp3d}{persp3d.default}
\keyword{dynamic}{persp3d}
\begin{Description}\relax
This function draws plots of surfaces over the
x-y plane. \code{persp3d} is a generic function.
\end{Description}
\begin{Usage}
\begin{verbatim}
persp3d(x, ...)

## Default S3 method:
persp3d(x = seq(0, 1, len = nrow(z)), y = seq(0, 1, len = ncol(z)),
    z, xlim = range(x), ylim = range(y), zlim = range(z, na.rm = TRUE),
    xlab = NULL, ylab = NULL, zlab = NULL, add = FALSE, aspect = !add, ...)
\end{verbatim}
\end{Usage}
\begin{Arguments}
\begin{ldescription}
\item[\code{x, y}] locations of grid lines at which the values in \code{z} are
measured.  These must be in ascending order.  By default, equally
spaced values from 0 to 1 are used.  If \code{x} is a \code{list},
its components \code{x\$x} and \code{x\$y} are used for \code{x}
and \code{y}, respectively.
\item[\code{z}] a matrix containing the values to be plotted.  
Note that \code{x} can be used instead of \code{z} for
convenience.
\item[\code{xlim, ylim, zlim}] x-, y-  and z-limits.  The plot is produced
so that the rectangular volume defined by these limits is visible.
\item[\code{xlab, ylab, zlab}] titles for the axes.  N.B. These must be
character strings; expressions are not accepted.  Numbers will be
coerced to character strings.
\item[\code{add}] whether to add the points to an existing plot.
\item[\code{aspect}] either a logical indicating whether to adjust the aspect ratio, or a new ratio
\item[\code{...}] additional material parameters to be passed to \code{\LinkA{surface3d}{surface3d}}
and \code{\LinkA{decorate3d}{decorate3d}}.
\end{ldescription}
\end{Arguments}
\begin{Details}\relax
This is similar to \code{\LinkA{persp}{persp}} with user interaction.  See \code{\LinkA{plot3d}{plot3d}}
for more details.
\end{Details}
\begin{Value}
This function is called for the side effect of drawing the plot.  A vector 
of shape IDs is returned.
\end{Value}
\begin{Author}\relax
Duncan Murdoch
\end{Author}
\begin{SeeAlso}\relax
\code{\LinkA{plot3d}{plot3d}}, \code{\LinkA{persp}{persp}}
\end{SeeAlso}
\begin{Examples}
\begin{ExampleCode}

# (1) The Obligatory Mathematical surface.
#     Rotated sinc function.

x <- seq(-10, 10, length= 30)
y <- x
f <- function(x,y) { r <- sqrt(x^2+y^2); 10 * sin(r)/r }
z <- outer(x, y, f)
z[is.na(z)] <- 1
open3d()
bg3d("white")
material3d(col="black")
persp3d(x, y, z, aspect=c(1, 1, 0.5), col = "lightblue",
        xlab = "X", ylab = "Y", zlab = "Sinc( r )")

# (2) Add to existing persp plot:

xE <- c(-10,10); xy <- expand.grid(xE, xE)
points3d(xy[,1], xy[,2], 6, col = 2, size = 3)
lines3d(x, y=10, z= 6 + sin(x), col = 3)

phi <- seq(0, 2*pi, len = 201)
r1 <- 7.725 # radius of 2nd maximum
xr <- r1 * cos(phi)
yr <- r1 * sin(phi)
lines3d(xr,yr, f(xr,yr), col = "pink", size = 2)

# (3) Visualizing a simple DEM model

z <- 2 * volcano        # Exaggerate the relief
x <- 10 * (1:nrow(z))   # 10 meter spacing (S to N)
y <- 10 * (1:ncol(z))   # 10 meter spacing (E to W)

open3d()
bg3d("slategray")
material3d(col="black")
persp3d(x, y, z, col = "green3", aspect="iso",
      axes = FALSE, box = FALSE)
\end{ExampleCode}
\end{Examples}

\HeaderA{plot3d}{3D Scatterplot}{plot3d}
\aliasA{decorate3d}{plot3d}{decorate3d}
\methaliasA{plot3d.default}{plot3d}{plot3d.default}
\methaliasA{plot3d.qmesh3d}{plot3d}{plot3d.qmesh3d}
\keyword{dynamic}{plot3d}
\begin{Description}\relax
Draws a 3D scatterplot.
\end{Description}
\begin{Usage}
\begin{verbatim}
plot3d(x, ...)
## Default S3 method:
plot3d(x, y, z,  
        xlab, ylab, zlab, type = "p", col,  
        size, radius,
        add = FALSE, aspect = !add, ...)
## S3 method for class 'qmesh3d':
plot3d(x, xlab = "x", ylab = "y", zlab = "z", type = c("shade", "wire", "dots"),
        add = FALSE, ...)       
decorate3d(xlim, ylim, zlim, 
        xlab = "x", ylab = "y", zlab = "z", 
        box = TRUE, axes = TRUE, main = NULL, sub = NULL,
        top = TRUE, aspect = FALSE, ...)
\end{verbatim}
\end{Usage}
\begin{Arguments}
\begin{ldescription}
\item[\code{x, y, z}] vectors of points to be plotted. Any reasonable way of defining the
coordinates is acceptable.  See the function \code{\LinkA{xyz.coords}{xyz.coords}}
for details.
\item[\code{xlab, ylab, zlab}] labels for the coordinates.
\item[\code{type}] For the default method, a single character indicating the type of item to plot.  
Supported types are: 'p' for points, 's' for spheres, 
'l' for lines, 'h' for line segments 
from \code{z=0}, and 'n' for nothing.  For the \code{qmesh3d} method, one of 
'shade', 'wire', or 'dots'.  Partial matching is used.

\item[\code{col}] the colour to be used for plotted items.
\item[\code{size}] the size for plotted items.
\item[\code{radius}] the radius of spheres:  see Details below.
\item[\code{add}] whether to add the points to an existing plot.
\item[\code{aspect}] either a logical indicating whether to adjust the aspect ratio, or a new ratio.
\item[\code{...}] additional parameters which will be passed to \code{\LinkA{par3d}{par3d}}, \code{\LinkA{material3d}{material3d}}
or \code{decorate3d}.
\item[\code{xlim, ylim, zlim}] limits to use for the coordinates.
\item[\code{box, axes}] whether to draw a box and axes.
\item[\code{main, sub}] main title and subtitle.
\item[\code{top}] whether to bring the window to the top when done.
\end{ldescription}
\end{Arguments}
\begin{Details}\relax
\code{plot3d} is a partial 3D analogue of plot.default.

Note that since \code{rgl} does not currently support
clipping, all points will be plotted, and \code{xlim}, \code{ylim}, and \code{zlim}
will only be used to increase the respective ranges.

Missing values in the data are skipped, as in standard graphics.

If \code{aspect} is \code{TRUE}, aspect ratios of \code{c(1,1,1)} are passed to
\code{\LinkA{aspect3d}{aspect3d}}.  If \code{FALSE}, no aspect adjustment is done.  In other
cases, the value is passed to \code{\LinkA{aspect3d}{aspect3d}}.

With \code{type = "s"}, spheres are drawn centered at the specified locations.
The radius may be controlled by \code{size} (specifying the size relative
to the plot display, with \code{size=1} giving a radius 
about 1/20 of the plot region) or \code{radius} (specifying it on the data scale
if an isometric aspect ratio is chosen, or on an average scale
if not).
\end{Details}
\begin{Value}
\code{plot3d} is called for the side effect of drawing the plot; a vector
of object IDs is returned.

\code{decorate3d} adds the usual decorations to a plot:  labels, axes, etc.
\end{Value}
\begin{Author}\relax
Duncan Murdoch
\end{Author}
\begin{SeeAlso}\relax
\code{\LinkA{plot.default}{plot.default}},  
\code{\LinkA{open3d}{open3d}}, \code{\LinkA{par3d}{par3d}}.
\end{SeeAlso}
\begin{Examples}
\begin{ExampleCode}
  open3d()
  x <- sort(rnorm(1000))
  y <- rnorm(1000)
  z <- rnorm(1000) + atan2(x,y)
  plot3d(x, y, z, col=rainbow(1000), size=2)
\end{ExampleCode}
\end{Examples}

\HeaderA{rgl.postscript}{export screenshot}{rgl.postscript}
\keyword{dynamic}{rgl.postscript}
\begin{Description}\relax
Saves the screenshot as PostScript file.
\end{Description}
\begin{Usage}
\begin{verbatim}
rgl.postscript( filename, fmt="eps" )
\end{verbatim}
\end{Usage}
\begin{Arguments}
\begin{ldescription}
\item[\code{filename}] full path to filename.
\item[\code{fmt}] export format, currently supported: ps, eps, tex, pdf, svg, pgf 
\end{ldescription}
\end{Arguments}
\begin{Details}\relax
Animations can be created in a loop modifying the scene and saving 
a screenshot to a file. (See example below)
\end{Details}
\begin{Author}\relax
Christophe Geuzaine / Albrecht Gebhardt
\end{Author}
\begin{References}\relax
GL2PS: an OpenGL to PostScript printing library by Christophe Geuzaine
\end{References}
\begin{SeeAlso}\relax
\code{\LinkA{rgl.viewpoint}{rgl.viewpoint}}, \code{\LinkA{rgl.snapshot}{rgl.snapshot}}
\end{SeeAlso}
\begin{Examples}
\begin{ExampleCode}

## Not run: 

#
# create a series of frames for an animation
#

rgl.open()
shade3d(oh3d(), color="red")
rgl.viewpoint(0,20)

for (i in 1:45) {
  rgl.viewpoint(i,20)
  filename <- paste("pic",formatC(i,digits=1,flag="0"),".eps",sep="") 
  rgl.postscript(filename, fmt="eps")
}

## End(Not run)

\end{ExampleCode}
\end{Examples}

\HeaderA{rgl.primitive}{add primitive set shape}{rgl.primitive}
\aliasA{rgl.lines}{rgl.primitive}{rgl.lines}
\aliasA{rgl.linestrips}{rgl.primitive}{rgl.linestrips}
\aliasA{rgl.points}{rgl.primitive}{rgl.points}
\aliasA{rgl.quads}{rgl.primitive}{rgl.quads}
\aliasA{rgl.triangles}{rgl.primitive}{rgl.triangles}
\keyword{dynamic}{rgl.primitive}
\begin{Description}\relax
Adds a shape node to the current scene
\end{Description}
\begin{Usage}
\begin{verbatim}
rgl.points(x, y = NULL, z = NULL, ... )
rgl.lines(x, y = NULL, z = NULL, ... )
rgl.linestrips(x, y = NULL, z = NULL, ...)
rgl.triangles(x, y = NULL, z = NULL, ... )
rgl.quads(x, y = NULL, z = NULL, ... )
\end{verbatim}
\end{Usage}
\begin{Arguments}
\begin{ldescription}
\item[\code{x, y, z}] coordinates.  Any reasonable way of defining the
coordinates is acceptable.  See the function \code{\LinkA{xyz.coords}{xyz.coords}}
for details.
\item[\code{ ... }] Material properties. See \code{\LinkA{rgl.material}{rgl.material}} for details.
\end{ldescription}
\end{Arguments}
\begin{Details}\relax
Adds a shape node to the scene. The appearance is defined by the material properties.
See \code{\LinkA{rgl.material}{rgl.material}} for details.
\end{Details}
\begin{Value}
Each primitive function returns the integer object ID of the shape that
was added to the scene.  These can be passed to \code{\LinkA{rgl.pop}{rgl.pop}}
to remove the object from the scene.
\end{Value}
\begin{SeeAlso}\relax
\code{\LinkA{rgl.material}{rgl.material}},
\code{\LinkA{rgl.spheres}{rgl.spheres}},
\code{\LinkA{rgl.texts}{rgl.texts}},
\code{\LinkA{rgl.surface}{rgl.surface}},
\code{\LinkA{rgl.sprites}{rgl.sprites}}
\end{SeeAlso}
\begin{Examples}
\begin{ExampleCode}
rgl.open()
rgl.points(rnorm(1000), rnorm(1000), rnorm(1000), color=heat.colors(1000), size=2)
\end{ExampleCode}
\end{Examples}

\HeaderA{qmesh3d}{3D Quadrangle Mesh objects}{qmesh3d}
\aliasA{cube3d}{qmesh3d}{cube3d}
\aliasA{dot3d}{qmesh3d}{dot3d}
\methaliasA{dot3d.qmesh3d}{qmesh3d}{dot3d.qmesh3d}
\aliasA{oh3d}{qmesh3d}{oh3d}
\aliasA{shade3d}{qmesh3d}{shade3d}
\methaliasA{shade3d.qmesh3d}{qmesh3d}{shade3d.qmesh3d}
\aliasA{wire3d}{qmesh3d}{wire3d}
\methaliasA{wire3d.qmesh3d}{qmesh3d}{wire3d.qmesh3d}
\keyword{dynamic}{qmesh3d}
\begin{Description}\relax
3D Quadrangle Mesh object creation and a collection of sample objects.
\end{Description}
\begin{Usage}
\begin{verbatim}
  qmesh3d(vertices, indices, homogeneous = TRUE, material = NULL)
  cube3d(trans = identityMatrix(), ...)  # cube object
  oh3d(trans = identityMatrix(), ...)    # an 'o' object
  
  dot3d(x, ...)   # draw dots at the vertices of an object
  ## S3 method for class 'qmesh3d':
  dot3d(x, override = TRUE, ...)
  wire3d(x, ...)  # draw a wireframe object
  ## S3 method for class 'qmesh3d':
  wire3d(x, override = TRUE, ...)
  shade3d(x, ...) # draw a shaded object
  ## S3 method for class 'qmesh3d':
  shade3d(x, override = TRUE, ...)
\end{verbatim}
\end{Usage}
\begin{Arguments}
\begin{ldescription}
\item[\code{x}] a qmesh3d object (class qmesh3d)
\item[\code{vertices}] 3- or 4-component vector of coordinates
\item[\code{indices}] 4-component vector of quad indices
\item[\code{homogeneous}] logical indicating if homogeneous (four component) coordinates are used.
\item[\code{material}] material properties for later rendering
\item[\code{trans}] transformation to apply to objects; see below for defaults
\item[\code{...}] additional rendering parameters
\item[\code{override}] should the parameters specified here override those stored in the object?
\end{ldescription}
\end{Arguments}
\begin{Details}\relax
The \code{cube3d} and \code{oh3d} objects optionally take a matrix transformation as 
an argument.  This transformation is applied to all vertices of the default shape.

The default is an identity transformation.  Use \code{par3d("userMatrix")} to render the
object vertically in the current user view.
\end{Details}
\begin{Value}
\code{qmesh3d}, \code{cube3d}, and \code{oh3d} return \code{qmesh3d} objects.

\code{dot3d}, \code{wire3d}, and \code{shade3d} are called for their side effect
of drawing an object into the scene; they return an object ID.
\end{Value}
\begin{SeeAlso}\relax
\code{\LinkA{r3d}{r3d}}, \code{\LinkA{par3d}{par3d}}
\end{SeeAlso}
\begin{Examples}
\begin{ExampleCode}

  # generate a quad mesh object

  vertices <- c( 
     -1.0, -1.0, 0, 1.0,
      1.0, -1.0, 0, 1.0,
      1.0,  1.0, 0, 1.0,
     -1.0,  1.0, 0, 1.0
  )
  indices <- c( 1, 2, 3, 4 )
  
  open3d()  
  wire3d( qmesh3d(vertices,indices) )
  
  # render 4 meshes vertically in the current view

  open3d()  
  bg3d("gray")
  l0 <- oh3d(tran = par3d("userMatrix"), color = "green" )
  shade3d( translate3d( l0, -6, 0, 0 ))
  l1 <- subdivision3d( l0 )
  shade3d( translate3d( l1 , -2, 0, 0 ), color="red", override = FALSE )
  l2 <- subdivision3d( l1 )
  shade3d( translate3d( l2 , 2, 0, 0 ), color="red", override = TRUE )
  l3 <- subdivision3d( l2 )
  shade3d( translate3d( l3 , 6, 0, 0 ), color="red" )
  
\end{ExampleCode}
\end{Examples}

\HeaderA{r3d}{Generic 3D interface}{r3d}
\keyword{dynamic}{r3d}
\begin{Description}\relax
Generic 3D interface for 3D rendering and computational geometry.
\end{Description}
\begin{Details}\relax
R3d is a design for an interface for 3d rendering and computation without dependency
on a specific rendering implementation. R3d includes a collection
of 3D objects and geometry algorithms.
All r3d interface functions are named \code{*3d}.  They represent generic functions that delegate 
to implementation functions.

The interface can be grouped into 8 categories: Scene Management, Primitive Shapes,
High-level Shapes, Geometry Objects, Visualization, Interaction, Transformation,
Subdivision.  

The rendering interface gives an abstraction to the underlying rendering model. It can
be grouped into four categories:    
\Enumerate{
\item[Scene Management:] A 3D scene consists of shapes, lights and background environment. 
\item[Primitive Shapes:] Generic primitive 3D graphics shapes such as points, lines, triangles, quadrangles and texts. 
\item[High-level Shapes:] Generic high-level 3D graphics shapes such as spheres, sprites and terrain.
\item[Interaction:] Generic interface to select points in 3D space using the pointer device.
}

In this package we include an implementation of r3d using the underlying \code{rgl.*} functions.

3D computation is supported through the use of object structures that live entirely in R.
\Enumerate{
\item[Geometry Objects:] Geometry and mesh objects allow to define high-level geometry for computational purpose such as quadrangle meshes (qmesh3d).
\item[Transformation:] Generic interface to transform 3d objects.
\item[Visualization:] Generic rendering of 3d objects such as dotted, wired or shaded.
\item[Computation:] Generic subdivision of 3d objects.
}

At present, there are two main practical differences between the r3d functions
and the \code{rgl.*} functions is that the r3d functions call
\code{\LinkA{open3d}{open3d}} if there is no device open, and the
\code{rgl.*} functions call \code{\LinkA{rgl.open}{rgl.open}}. By default
\code{\LinkA{open3d}{open3d}} sets the initial orientation of the coordinate
system in 'world coordinates', i.e. a right-handed coordinate system
in which the x-axis increasingfrom left to right, the y-axis
increases with depth into the scene, and the z-axis increases from
bottom to top of the screen.  \code{rgl.*} functions, on the other
hand, use a right-handed coordinate system similar to that used in
OpenGL.  The x-axis matches that of r3d, but the y-axis increases
from bottom to top, and the z-axis decreases with depth into the
scene.  Since the user can manipulate the scene, either system can
be rotated into the other one.  

The r3d functions also preserve the \code{rgl.material} setting across
calls (except for texture elements, in the current implementation), whereas
the \code{rgl.*} functions leave it as set by the last call.

The example code below illustrates the two coordinate systems.
\end{Details}
\begin{SeeAlso}\relax
\code{\LinkA{points3d}{points3d}}
\code{\LinkA{lines3d}{lines3d}}
\code{\LinkA{segments3d}{segments3d}}
\code{\LinkA{triangles3d}{triangles3d}}
\code{\LinkA{quads3d}{quads3d}}
\code{\LinkA{text3d}{text3d}}
\code{\LinkA{spheres3d}{spheres3d}}
\code{\LinkA{sprites3d}{sprites3d}}
\code{\LinkA{terrain3d}{terrain3d}}
\code{\LinkA{select3d}{select3d}}
\code{\LinkA{dot3d}{dot3d}}
\code{\LinkA{wire3d}{wire3d}}
\code{\LinkA{shade3d}{shade3d}}
\code{\LinkA{transform3d}{transform3d}}
\code{\LinkA{rotate3d}{rotate3d}}
\code{\LinkA{subdivision3d}{subdivision3d}}
\code{\LinkA{qmesh3d}{qmesh3d}}
\code{\LinkA{cube3d}{cube3d}}
\code{\LinkA{rgl}{rgl}}
\end{SeeAlso}
\begin{Examples}
\begin{ExampleCode}
    
     x <- c(0,1,0,0)
     y <- c(0,0,1,0)
     z <- c(0,0,0,1)
     labels <- c("Origin", "X", "Y", "Z")
     i <- c(1,2,1,3,1,4)

     rgl.open()
     rgl.texts(x,y,z,labels)
     rgl.texts(1,1,1,"rgl.* coordinates")
     rgl.lines(x[i],y[i],z[i])

     open3d()
     text3d(x,y,z,labels)
     text3d(1,1,1,"*3d coordinates")
     segments3d(x[i],y[i],z[i])
\end{ExampleCode}
\end{Examples}

\HeaderA{rgl-internal}{Internal rgl functions and data}{rgl.Rdash.internal}
\aliasA{cube3d.ib}{rgl-internal}{cube3d.ib}
\aliasA{cube3d.vb}{rgl-internal}{cube3d.vb}
\aliasA{dev3d}{rgl-internal}{dev3d}
\aliasA{edgeindex}{rgl-internal}{edgeindex}
\aliasA{edgemap}{rgl-internal}{edgemap}
\aliasA{oh3d.ib}{rgl-internal}{oh3d.ib}
\aliasA{oh3d.vb}{rgl-internal}{oh3d.vb}
\aliasA{rgl.attr}{rgl-internal}{rgl.attr}
\aliasA{rgl.bool}{rgl-internal}{rgl.bool}
\aliasA{rgl.clamp}{rgl-internal}{rgl.clamp}
\aliasA{rgl.color}{rgl-internal}{rgl.color}
\aliasA{rgl.enum}{rgl-internal}{rgl.enum}
\methaliasA{rgl.enum.fogtype}{rgl-internal}{rgl.enum.fogtype}
\methaliasA{rgl.enum.gl2ps}{rgl-internal}{rgl.enum.gl2ps}
\methaliasA{rgl.enum.halign}{rgl-internal}{rgl.enum.halign}
\methaliasA{rgl.enum.nodetype}{rgl-internal}{rgl.enum.nodetype}
\methaliasA{rgl.enum.pixfmt}{rgl-internal}{rgl.enum.pixfmt}
\methaliasA{rgl.enum.polymode}{rgl-internal}{rgl.enum.polymode}
\methaliasA{rgl.enum.primtype}{rgl-internal}{rgl.enum.primtype}
\methaliasA{rgl.enum.texmagfilter}{rgl-internal}{rgl.enum.texmagfilter}
\methaliasA{rgl.enum.texminfilter}{rgl-internal}{rgl.enum.texminfilter}
\methaliasA{rgl.enum.textype}{rgl-internal}{rgl.enum.textype}
\aliasA{rgl.mcolor}{rgl-internal}{rgl.mcolor}
\aliasA{rgl.numeric}{rgl-internal}{rgl.numeric}
\aliasA{rgl.nvertex}{rgl-internal}{rgl.nvertex}
\aliasA{rgl.range}{rgl-internal}{rgl.range}
\aliasA{rgl.select}{rgl-internal}{rgl.select}
\aliasA{rgl.selectstate}{rgl-internal}{rgl.selectstate}
\aliasA{rgl.setselectstate}{rgl-internal}{rgl.setselectstate}
\aliasA{rgl.vertex}{rgl-internal}{rgl.vertex}
\keyword{internal}{rgl-internal}
\begin{Description}\relax
internal rgl functions
\end{Description}
\begin{Usage}
\begin{verbatim}
rgl.bool(x)
rgl.numeric(x)
rgl.range(x, low, high)
rgl.vertex(x, y = NULL, z = NULL)
rgl.nvertex(vertex)
rgl.color(color)
rgl.mcolor(colors)
rgl.clamp(value, low, high)
rgl.attr(vattr, nvertex) 
rgl.enum(name, ..., multi = FALSE)
rgl.enum.gl2ps(postscripttype) 
rgl.enum.nodetype(type)
rgl.enum.pixfmt(fmt)
rgl.enum.polymode(mode)
rgl.enum.textype(textype)
rgl.enum.fogtype(fogtype)
rgl.enum.primtype(primtype)
rgl.enum.halign(halign)
rgl.enum.texmagfilter(magfiltertype)
rgl.enum.texminfilter(minfiltertype)
rgl.select(button = c("left", "middle", "right"))
rgl.selectstate()
rgl.setselectstate(state)
edgemap(size)
edgeindex(from,to,size,row,col)
\end{verbatim}
\end{Usage}
\begin{Details}\relax
These are not to be called by the user.
\end{Details}

\HeaderA{rgl-package}{3D visualization device system}{rgl.Rdash.package}
\aliasA{rgl}{rgl-package}{rgl}
\aliasA{rgl.close}{rgl-package}{rgl.close}
\aliasA{rgl.cur}{rgl-package}{rgl.cur}
\aliasA{rgl.open}{rgl-package}{rgl.open}
\aliasA{rgl.quit}{rgl-package}{rgl.quit}
\aliasA{rgl.set}{rgl-package}{rgl.set}
\keyword{dynamic}{rgl-package}
\begin{Description}\relax
3D real-time rendering device-driver system
\end{Description}
\begin{Usage}
\begin{verbatim}
rgl.open()     # open new device
rgl.close()    # close current device
rgl.cur()      # returns active device ID
rgl.set(which) # set device as active
rgl.quit()     # shutdown rgl device system
\end{verbatim}
\end{Usage}
\begin{Arguments}
\begin{ldescription}
\item[\code{which}] device ID
\end{ldescription}
\end{Arguments}
\begin{Details}\relax
RGL is a 3D real-time rendering device driver system for R.
Multiple devices are managed at a time, where one has the current focus
that receives instructions from the R command-line.
The device design is oriented towards the R device metaphor. If you send
scene management instructions, and there's no device open, it will be opened
automatically.
Opened devices automatically get the current device focus. The focus may be
changed by using rgl.set().
rgl.quit() shuts down the rgl subsystem and all open devices, 
detaches the package including the shared library and additional system libraries.
\end{Details}
\begin{SeeAlso}\relax
\code{\LinkA{rgl.clear}{rgl.clear}}, 
\code{\LinkA{rgl.pop}{rgl.pop}},
\code{\LinkA{rgl.viewpoint}{rgl.viewpoint}},
\code{\LinkA{rgl.light}{rgl.light}},
\code{\LinkA{rgl.bg}{rgl.bg}},
\code{\LinkA{rgl.bbox}{rgl.bbox}},
\code{\LinkA{rgl.points}{rgl.points}},
\code{\LinkA{rgl.lines}{rgl.lines}},
\code{\LinkA{rgl.triangles}{rgl.triangles}},
\code{\LinkA{rgl.quads}{rgl.quads}},
\code{\LinkA{rgl.texts}{rgl.texts}},
\code{\LinkA{rgl.surface}{rgl.surface}},
\code{\LinkA{rgl.spheres}{rgl.spheres}},
\code{\LinkA{rgl.sprites}{rgl.sprites}},
\code{\LinkA{rgl.snapshot}{rgl.snapshot}}
\end{SeeAlso}
\begin{Examples}
\begin{ExampleCode}
example(surface3d)
example(plot3d)
\end{ExampleCode}
\end{Examples}

\HeaderA{rgl.bringtotop}{Assign focus to an RGL window}{rgl.bringtotop}
\keyword{dynamic}{rgl.bringtotop}
\begin{Description}\relax
'rgl.bringtotop' brings the current RGL window to the
front of the window stack (and gives it focus).
\end{Description}
\begin{Usage}
\begin{verbatim}
rgl.bringtotop(stay = FALSE)
\end{verbatim}
\end{Usage}
\begin{Arguments}
\begin{ldescription}
\item[\code{stay}] whether to make the window stay on top.
\end{ldescription}
\end{Arguments}
\begin{Details}\relax
If \code{stay} is \code{TRUE}, then the window will stay on top
of normal windows.
\end{Details}
\begin{Note}\relax
not completely implemented for X11 graphics (stay not implemented;
window managers such as KDE may block this action
(set "Focus stealing prevention level" to None in Control Center/Window Behavior/Advanced)).
Not currently implemented under OS/X.
\end{Note}
\begin{Author}\relax
Ming Chen/Duncan Murdoch
\end{Author}
\begin{SeeAlso}\relax
\code{\LinkA{bringToTop}{bringToTop}}
\end{SeeAlso}
\begin{Examples}
\begin{ExampleCode}
rgl.open()
rgl.points(rnorm(1000), rnorm(1000), rnorm(1000), color=heat.colors(1000), size=2)
rgl.bringtotop(stay = TRUE)    
\end{ExampleCode}
\end{Examples}

\HeaderA{rgl.user2window}{Convert between rgl user and window coordinates}{rgl.user2window}
\aliasA{rgl.projection}{rgl.user2window}{rgl.projection}
\aliasA{rgl.window2user}{rgl.user2window}{rgl.window2user}
\keyword{dynamic}{rgl.user2window}
\begin{Description}\relax
This function converts from 3-dimensional user coordinates
to 3-dimensional window coordinates.
\end{Description}
\begin{Usage}
\begin{verbatim}
rgl.user2window(x, y = NULL, z = NULL, projection = rgl.projection())
rgl.window2user(x, y = NULL, z = 0, projection = rgl.projection())
rgl.projection()
\end{verbatim}
\end{Usage}
\begin{Arguments}
\begin{ldescription}
\item[\code{x, y, z}] Input coordinates.  Any reasonable way of defining the
coordinates is acceptable.  See the function \code{\LinkA{xyz.coords}{xyz.coords}}
for details.
\item[\code{projection}] The rgl projection to use 
\end{ldescription}
\end{Arguments}
\begin{Details}\relax
These functions convert between user coordinates and window coordinates.

Window coordinates run from 0 to 1 in X, Y, and Z.  X runs from 0 on the
left to 1 on the right; Y runs from 0 at the bottom to 1 at the top;
Z runs from 0 foremost to 1 in the background.  \code{rgl} does not currently
display vertices plotted outside of this range, but in normal circumstances will automatically resize the
display to show them.  In the example below this has been suppressed.
\end{Details}
\begin{Value}
The coordinate conversion functions produce a matrix with columns corresponding 
to the X, Y, and Z coordinates.

\code{rgl.projection()} returns a list containing the model matrix, projection matrix
and viewport.  See \code{\LinkA{par3d}{par3d}} for more details.
\end{Value}
\begin{Author}\relax
Ming Chen / Duncan Murdoch
\end{Author}
\begin{SeeAlso}\relax
\code{\LinkA{select3d}{select3d}}
\end{SeeAlso}
\begin{Examples}
\begin{ExampleCode}
open3d()
points3d(rnorm(100), rnorm(100), rnorm(100))
if (interactive() || !.Platform$OS=="unix") {
# Calculate a square in the middle of the display and plot it
square <- rgl.window2user(c(0.25, 0.25, 0.75, 0.75, 0.25), 
                          c(0.25, 0.75, 0.75, 0.25, 0.25), 0.5)
par3d(ignoreExtent = TRUE)
lines3d(square)
par3d(ignoreExtent = FALSE)
}
\end{ExampleCode}
\end{Examples}

\HeaderA{scene}{scene management}{scene}
\aliasA{clear3d}{scene}{clear3d}
\aliasA{pop3d}{scene}{pop3d}
\aliasA{rgl.clear}{scene}{rgl.clear}
\aliasA{rgl.ids}{scene}{rgl.ids}
\aliasA{rgl.pop}{scene}{rgl.pop}
\keyword{dynamic}{scene}
\begin{Description}\relax
Clear shapes, lights, bbox
\end{Description}
\begin{Usage}
\begin{verbatim}
clear3d( type = c("shapes", "bboxdeco") ) 
rgl.clear( type = "shapes" )
pop3d( ... )
rgl.pop( type = "shapes", id = 0 )  
rgl.ids( type = "shapes" )
\end{verbatim}
\end{Usage}
\begin{Arguments}
\begin{ldescription}
\item[\code{type}] Select subtype(s):
\describe{
\item["shapes"] shape stack
\item["lights"] light stack
\item["bboxdeco"] bounding box
\item["viewpoint"] viewpoint
\item["all"] all of the above
}

\item[\code{id}] vector of ID numbers of items to remove
\item[\code{...}] generic arguments passed through to RGL-specific (or other) functions
\end{ldescription}
\end{Arguments}
\begin{Details}\relax
RGL holds two stacks. One is for shapes and the other is for lights. 
\code{clear3d} and \code{rgl.clear} clear the specified stack, or restore
the defaults for the bounding box (not visible) or viewpoint.  By default
with \code{id=0} \code{rgl.pop} removes 
the top-most (last added) node on the shape stack.  The \code{id} argument
may be used to specify arbitrary item(s) to remove from the specified stack.

\code{rgl.ids} returns a dataframe containing the IDs in the currently active
rgl window, along with an indicator of their type.

For convenience, \code{type="shapes"} and 
\code{id=1} signifies the bounding box.

Note that clearing the light stack leaves the scene in darkness; it should normally
be followed by a call to \code{\LinkA{rgl.light}{rgl.light}} or \code{\LinkA{light3d}{light3d}}.
\end{Details}
\begin{SeeAlso}\relax
\code{\LinkA{rgl}{rgl}}
\code{\LinkA{rgl.bbox}{rgl.bbox}}
\code{\LinkA{rgl.light}{rgl.light}}
\end{SeeAlso}
\begin{Examples}
\begin{ExampleCode}
  x <- rnorm(100)
  y <- rnorm(100)
  z <- rnorm(100)
  p <- plot3d(x, y, z, type='s')
  rgl.ids()
  lines3d(x, y, z)
  rgl.ids()
  if (interactive()) {
    readline("Hit enter to change spheres")
    rgl.pop(id = p[c("data", "box.lines")])
    spheres3d(x, y, z, col="red", radius=1/5)
    box3d()
  }
\end{ExampleCode}
\end{Examples}

\HeaderA{select3d}{Select a rectangle in an RGL scene}{select3d}
\aliasA{rgl.select3d}{select3d}{rgl.select3d}
\keyword{dynamic}{select3d}
\begin{Description}\relax
This function allows the user to use the mouse to
select a region in an RGL scene.
\end{Description}
\begin{Usage}
\begin{verbatim}
rgl.select3d(button = c("left", "middle", "right"))
select3d(...)
\end{verbatim}
\end{Usage}
\begin{Arguments}
\begin{ldescription}
\item[\code{ button }] Which button to use for selection.
\item[\code{ ... }] Button argument to pass to \code{rgl.select3d}
\end{ldescription}
\end{Arguments}
\begin{Details}\relax
This function selects 3-dimensional regions by allowing the
user to use a mouse to draw a rectangle showing
the projection of the region onto the screen.  It returns
a function which tests points for inclusion in the selected region.

If the scene is later moved or rotated, the selected region will 
remain the same, no longer corresponding to a rectangle on the screen.
\end{Details}
\begin{Value}
Returns a function \code{f(x,y,z)} which tests whether each
of the points \code{(x,y,z)} is in the selected region, returning
a logical vector.  This function accepts input in a wide
variety of formats as it uses \code{\LinkA{xyz.coords}{xyz.coords}} 
to interpret its parameters.
\end{Value}
\begin{Author}\relax
Ming Chen / Duncan Murdoch
\end{Author}
\begin{SeeAlso}\relax
\code{\LinkA{locator}{locator}}
\end{SeeAlso}
\begin{Examples}
\begin{ExampleCode}

# Allow the user to select some points, and then redraw them
# in a different color

if (interactive()) {
 x <- rnorm(1000)
 y <- rnorm(1000)
 z <- rnorm(1000)
 open3d()
 points3d(x,y,z,size=2)
 f <- select3d()
 keep <- f(x,y,z)
 rgl.pop()
 points3d(x[keep],y[keep],z[keep],size=2,color='red')
 points3d(x[!keep],y[!keep],z[!keep],size=2)
}
\end{ExampleCode}
\end{Examples}

\HeaderA{rgl.snapshot}{export screenshot}{rgl.snapshot}
\aliasA{snapshot3d}{rgl.snapshot}{snapshot3d}
\keyword{dynamic}{rgl.snapshot}
\begin{Description}\relax
Saves the screenshot as png file.
\end{Description}
\begin{Usage}
\begin{verbatim}
rgl.snapshot( filename, fmt="png" )
snapshot3d( ... )
\end{verbatim}
\end{Usage}
\begin{Arguments}
\begin{ldescription}
\item[\code{filename}] full path to filename.
\item[\code{fmt}] image export format, currently supported: png 
\item[\code{...}] arguments to pass to \code{rgl.snapshot} 
\end{ldescription}
\end{Arguments}
\begin{Details}\relax
Animations can be created in a loop modifying the scene and saving 
each screenshot to a file. Various graphics programs (e.g. ImageMagick)
can put these together into a single animation. (See example below)
\end{Details}
\begin{SeeAlso}\relax
\code{\LinkA{rgl.viewpoint}{rgl.viewpoint}}
\end{SeeAlso}
\begin{Examples}
\begin{ExampleCode}

## Not run: 

#
# create animation
#

shade3d(oh3d(), color="red")
rgl.viewpoint(0,20)

setwd(tempdir())
for (i in 1:45) {
  rgl.viewpoint(i,20)
  filename <- paste("pic",formatC(i,digits=1,flag="0"),".png",sep="")
  rgl.snapshot(filename)
}
## Now run ImageMagick command:
##    convert -delay 10 *.png -loop 0 pic.gif
## End(Not run)

\end{ExampleCode}
\end{Examples}

\HeaderA{spheres}{add sphere set shape}{spheres}
\aliasA{rgl.spheres}{spheres}{rgl.spheres}
\aliasA{spheres3d}{spheres}{spheres3d}
\keyword{dynamic}{spheres}
\begin{Description}\relax
Adds a sphere set shape node to the scene
\end{Description}
\begin{Usage}
\begin{verbatim}
spheres3d(x, y = NULL, z = NULL, radius = 1, ...)
rgl.spheres(x, y = NULL, z = NULL, radius, ...)
\end{verbatim}
\end{Usage}
\begin{Arguments}
\begin{ldescription}
\item[\code{x, y, z}] Numeric vector of point coordinates corresponding to
the center of each sphere.  Any reasonable way of defining the
coordinates is acceptable.  See the function \code{\LinkA{xyz.coords}{xyz.coords}}
for details.
\item[\code{radius}] Vector or single value defining the sphere radius/radii
\item[\code{ ... }] Material properties. See \code{\LinkA{rgl.material}{rgl.material}} for details.
\end{ldescription}
\end{Arguments}
\begin{Details}\relax
If a non-isometric aspect ratio is chosen, these functions will still draw
objects that appear to the viewer to be spheres.  Use \code{\LinkA{ellipse3d}{ellipse3d}} 
to draw shapes that are spherical in the data scale.

When the scale is not isometric, the radius is measured in an average scale.
Note that the bounding box calculation is always done assuming an 
isometric scale, so in this case it is inaccurate:  the extent of axes 
with \code{scale < 1} is underestimated, and that of axes with 
\code{scale > 1} is overestimated.

If any coordinate or radius is \code{NA}, the sphere is not plotted.
\end{Details}
\begin{Value}
A shape ID of the spheres object is returned.
\end{Value}
\begin{SeeAlso}\relax
\code{\LinkA{rgl.material}{rgl.material}}, \code{\LinkA{aspect3d}{aspect3d}} for setting non-isometric scales
\end{SeeAlso}
\begin{Examples}
\begin{ExampleCode}
open3d()
spheres3d(rnorm(10), rnorm(10), rnorm(10), radius=runif(10), color=rainbow(10))
\end{ExampleCode}
\end{Examples}

\HeaderA{sprites}{add sprite set shape}{sprites}
\aliasA{particles3d}{sprites}{particles3d}
\aliasA{rgl.sprites}{sprites}{rgl.sprites}
\aliasA{sprites3d}{sprites}{sprites3d}
\keyword{dynamic}{sprites}
\begin{Description}\relax
Adds a sprite set shape node to the scene.
\end{Description}
\begin{Usage}
\begin{verbatim}
sprites3d(x, y = NULL, z = NULL, radius = 1, ...)
particles3d(x, y = NULL, z = NULL, radius = 1, ...)
rgl.sprites(x, y = NULL, z = NULL, radius = 1, ...)
\end{verbatim}
\end{Usage}
\begin{Arguments}
\begin{ldescription}
\item[\code{ x, y, z }] point coordinates.  Any reasonable way of defining the
coordinates is acceptable.  See the function \code{\LinkA{xyz.coords}{xyz.coords}}
for details.
\item[\code{ radius }] 
\item[\code{ ... }] material properties, texture mapping is supported
\end{ldescription}
\end{Arguments}
\begin{Details}\relax
Sprites are rectangle planes that are directed towards the viewpoint.
Their primary use is for fast (and faked) atmospherical effects, e.g. particles and clouds
using alpha blended textures.
Particles are Sprites using an alpha-blended particle texture giving
the illusion of clouds and gasses.

If any coordinate is \code{NA}, the sprite is not plotted.
\end{Details}
\begin{Value}
These functions are called for the side effect of displaying the sprites.
The shape ID of the displayed object is returned.
\end{Value}
\begin{SeeAlso}\relax
\code{\LinkA{rgl.material}{rgl.material}}
\end{SeeAlso}
\begin{Examples}
\begin{ExampleCode}
open3d()
particles3d( rnorm(100), rnorm(100), rnorm(100), color=rainbow(100) )
# is the same as
sprites3d( rnorm(100), rnorm(100), rnorm(100), color=rainbow(100),
  lit=FALSE, alpha=.2,
  textype="alpha", texture=system.file("textures/particle.png", package="rgl") )
\end{ExampleCode}
\end{Examples}

\HeaderA{subdivision3d}{generic subdivision surface method}{subdivision3d}
\aliasA{deform.qmesh3d}{subdivision3d}{deform.qmesh3d}
\aliasA{divide.qmesh3d}{subdivision3d}{divide.qmesh3d}
\aliasA{normalize.qmesh3d}{subdivision3d}{normalize.qmesh3d}
\methaliasA{subdivision3d.qmesh3d}{subdivision3d}{subdivision3d.qmesh3d}
\keyword{dynamic}{subdivision3d}
\begin{Description}\relax
The Subdivision surface algorithm divide and refine (deform) a
given mesh recursively to certain degree (depth).
The qmesh3d algorithm consists of two stages: divide and deform.
The divide step generates for each quad four new quads, the deform
step drags the points (refinement step).
\end{Description}
\begin{Usage}
\begin{verbatim}
  subdivision3d( x,  ...)
  subdivision3d.qmesh3d( x, depth=1, normalize=FALSE, deform=TRUE, ... )
  divide.qmesh3d(mesh,  vb=mesh$vb, ib=mesh$ib )
  normalize.qmesh3d(mesh)
  deform.qmesh3d(mesh,vb=mesh$vb,ib=mesh$ib)
\end{verbatim}
\end{Usage}
\begin{Arguments}
\begin{ldescription}
\item[\code{x}] 3d geometry mesh
\item[\code{mesh}] 3d geometry mesh
\item[\code{depth}] recursion depth
\item[\code{normalize}] normalize qmesh3d coordinates after division if deform is TRUE otherwise
\item[\code{deform}] deform mesh after normalization
\item[\code{ib}] vector of indices (i.e., which sets of vertices to combine
into quads); 4xn matrix or equivalent vector
\item[\code{vb}] vector of vertices: 4xn matrix (rows x,y,z,h) or equivalent
vector, where h indicates scaling of each plotted quad
\item[\code{...}] other arguments (unused)
\end{ldescription}
\end{Arguments}
\begin{Details}\relax
Generic subdivision surface method. Currently there exists
an algorithm that can be applied on qmesh3d objects.
\end{Details}
\begin{SeeAlso}\relax
\code{\LinkA{r3d}{r3d}}
\code{\LinkA{qmesh3d}{qmesh3d}}
\end{SeeAlso}
\begin{Examples}
\begin{ExampleCode}
  open3d()
  shade3d( subdivision3d( cube3d(), depth=3 ), color="red", alpha=0.5 )
\end{ExampleCode}
\end{Examples}

\HeaderA{rgl.surface}{add height-field surface shape}{rgl.surface}
\keyword{dynamic}{rgl.surface}
\begin{Description}\relax
Adds a surface to the current scene. The surface is defined by 
a matrix defining the height of each grid point and two vectors
defining the grid.
\end{Description}
\begin{Usage}
\begin{verbatim}
rgl.surface(x, z, y, coords=1:3, ...)
\end{verbatim}
\end{Usage}
\begin{Arguments}
\begin{ldescription}
\item[\code{ x }] values corresponding to rows of \code{y}

\item[\code{ y }] matrix of height values

\item[\code{ z }] values corresponding to columns of \code{y}

\item[\code{ coords }] See details

\item[\code{ ... }] Material and texture properties. See \code{\LinkA{rgl.material}{rgl.material}} for details.
\end{ldescription}
\end{Arguments}
\begin{Details}\relax
Adds a surface mesh to the current scene. The surface is defined by 
the matrix of height values in \code{y}, with rows corresponding 
to the values in \code{x} and columns corresponding to the values in 
\code{z}.

The \code{coords} parameter can be used to change the geometric
interpretation of \code{x}, \code{y}, and \code{z}.  The first entry 
of \code{coords} indicates which coordinate (\code{1=X}, 
\code{2=Y}, \code{3=Z}) corresponds to the \code{x} parameter.
Similarly the second entry corresponds to the \code{y} parameter,
and the third entry to the \code{z} parameter.  In this way 
surfaces may be defined over any coordinate plane.

\code{rgl.surface} always draws the surface with the `front' upwards
(i.e. towards higher \code{y} values).  This can be used to render
the top and bottom differently; see \code{\LinkA{rgl.material}{rgl.material}} and
the example below.

\code{NA} values in the height matrix are not drawn.
\end{Details}
\begin{Value}
The object ID of the displayed surface is returned invisibly.
\end{Value}
\begin{SeeAlso}\relax
\code{\LinkA{rgl.material}{rgl.material}}, \code{\LinkA{surface3d}{surface3d}}, \code{\LinkA{terrain3d}{terrain3d}}
\end{SeeAlso}
\begin{Examples}
\begin{ExampleCode}

#
# volcano example taken from "persp"
#

data(volcano)

y <- 2 * volcano        # Exaggerate the relief

x <- 10 * (1:nrow(y))   # 10 meter spacing (S to N)
z <- 10 * (1:ncol(y))   # 10 meter spacing (E to W)

ylim <- range(y)
ylen <- ylim[2] - ylim[1] + 1

colorlut <- terrain.colors(ylen) # height color lookup table

col <- colorlut[ y-ylim[1]+1 ] # assign colors to heights for each point

rgl.open()
rgl.surface(x, z, y, color=col, back="lines")

\end{ExampleCode}
\end{Examples}

\HeaderA{surface3d}{add height-field surface shape}{surface3d}
\aliasA{terrain3d}{surface3d}{terrain3d}
\keyword{dynamic}{surface3d}
\begin{Description}\relax
Adds a surface to the current scene. The surface is defined by 
a matrix defining the height of each grid point and two vectors
defining the grid.
\end{Description}
\begin{Usage}
\begin{verbatim}
surface3d(x, y, z, ...)
terrain3d(x, y, z, ...)
\end{verbatim}
\end{Usage}
\begin{Arguments}
\begin{ldescription}
\item[\code{ x }] values corresponding to rows of \code{z}

\item[\code{ y }] values corresponding to the columns of \code{z}

\item[\code{ z }] matrix of heights

\item[\code{ ... }] Material and texture properties. See \code{\LinkA{rgl.material}{rgl.material}} for details.
\end{ldescription}
\end{Arguments}
\begin{Details}\relax
Adds a surface mesh to the current scene. The surface is defined by 
the matrix of height values in \code{z}, with rows corresponding 
to the values in \code{x} and columns corresponding to the values in 
\code{y}.  This is the same parametrization as used in \code{\LinkA{persp}{persp}}.

\code{surface3d} always draws the surface with the `front' upwards
(i.e. towards higher \code{z} values).  This can be used to render
the top and bottom differently; see \code{\LinkA{rgl.material}{rgl.material}} and
the example below.

For more flexibility in defining the surface, use \code{\LinkA{rgl.surface}{rgl.surface}}.

\code{surface3d} and \code{terrain3d} are synonyms.
\end{Details}
\begin{SeeAlso}\relax
\code{\LinkA{rgl.material}{rgl.material}}, \code{\LinkA{rgl.surface}{rgl.surface}}, \code{\LinkA{persp}{persp}}
\end{SeeAlso}
\begin{Examples}
\begin{ExampleCode}

#
# volcano example taken from "persp"
#

data(volcano)

z <- 2 * volcano        # Exaggerate the relief

x <- 10 * (1:nrow(z))   # 10 meter spacing (S to N)
y <- 10 * (1:ncol(z))   # 10 meter spacing (E to W)

zlim <- range(y)
zlen <- zlim[2] - zlim[1] + 1

colorlut <- terrain.colors(zlen) # height color lookup table

col <- colorlut[ z-zlim[1]+1 ] # assign colors to heights for each point

open3d()
surface3d(x, y, z, color=col, back="lines")

\end{ExampleCode}
\end{Examples}

\HeaderA{texts}{add text}{texts}
\aliasA{rgl.texts}{texts}{rgl.texts}
\aliasA{text3d}{texts}{text3d}
\aliasA{texts3d}{texts}{texts3d}
\keyword{dynamic}{texts}
\begin{Description}\relax
Adds text to the scene. The text is positioned in 3D space. 
A bitmap font which is always oriented towards the camera is used.
\end{Description}
\begin{Usage}
\begin{verbatim}
rgl.texts(x, y = NULL, z = NULL, text, adj = 0.5, justify, ...)
text3d(x, y = NULL, z = NULL, texts, adj = 0.5, justify, ...) 
texts3d(x, y = NULL, z = NULL, texts, adj = 0.5, justify, ...) 
\end{verbatim}
\end{Usage}
\begin{Arguments}
\begin{ldescription}
\item[\code{x, y, z}] point coordinates.  Any reasonable way of defining the
coordinates is acceptable.  See the function \code{\LinkA{xyz.coords}{xyz.coords}}
for details.
\item[\code{text}] text character vector to draw
\item[\code{texts}] text character vector to draw
\item[\code{adj}] one value specifying the horizontal adjustment 
\item[\code{justify}] (deprecated, please use adj instead) character string
specifying the horizontal adjustment; options are \code{"left"}, \code{"right"},
\code{"center"}.

\item[\code{ ... }] Material properties. See \code{\LinkA{rgl.material}{rgl.material}} for details.
\end{ldescription}
\end{Arguments}
\begin{Details}\relax
The \code{adj} parameter determines the position of the text relative to the 
specified coordinate.  Use \code{adj = 0} to place the left bottom corner at
\code{(x,y,z)}, \code{adj = 0.5} to center the text there, and \code{adj = 1}
to put the right bottom corner there.

\code{text3d} and \code{texts3d} draw text using the \LinkA{r3d}{r3d}
conventions. These are synonyms; the former is singular to be
consistent with the classic 2-D graphics functions, and the latter is
plural to be consistent with all the other graphics primitives.  Take
your choice!

If any coordinate or text is \code{NA}, that text is not plotted.
\end{Details}
\begin{Value}
The object ID of the text object is returned invisibly.
\end{Value}
\begin{SeeAlso}\relax
\code{\LinkA{r3d}{r3d}}
\end{SeeAlso}
\begin{Examples}
\begin{ExampleCode}
open3d()
text3d(rnorm(10)*100,rnorm(10)*100,rnorm(10)*100,text=1:10,adj = 0.5, 
       color=heat.colors(10))
\end{ExampleCode}
\end{Examples}

\HeaderA{viewpoint}{Set up viewpoint}{viewpoint}
\aliasA{rgl.viewpoint}{viewpoint}{rgl.viewpoint}
\aliasA{view3d}{viewpoint}{view3d}
\keyword{dynamic}{viewpoint}
\begin{Description}\relax
Set the viewpoint orientation.
\end{Description}
\begin{Usage}
\begin{verbatim}
view3d( theta = 0, phi = 15, ...)
rgl.viewpoint( theta = 0, phi = 15, fov = 60, zoom = 1, scale = par3d("scale"), 
               interactive = TRUE, userMatrix )
\end{verbatim}
\end{Usage}
\begin{Arguments}
\begin{ldescription}
\item[\code{theta,phi}] polar coordinates
\item[\code{...}] additional parameters to pass to \code{rgl.viewpoint}
\item[\code{fov}] field-of-view angle
\item[\code{zoom}] zoom factor
\item[\code{scale}] real length 3 vector specifying the rescaling to apply to each axis
\item[\code{interactive}] logical, specifying if interactive navigation is allowed
\item[\code{userMatrix}] 4x4 matrix specifying user point of view
\end{ldescription}
\end{Arguments}
\begin{Details}\relax
The viewpoint can be set in an orbit around the data model, using the polar coordinates \code{\\theta}
and \code{phi}.  Alternatively, it can be set in a completely general way using the 4x4 matrix
\code{userMatrix}.  If \code{userMatrix} is specified, \code{theta} and \code{phi} are ignored.

The pointing device of your graphics user-interface can also be used to 
set the viewpoint interactively. With the pointing device the buttons are by default set as follows:

\Itemize{
\item[left] adjust viewpoint position
\item[middle] adjust field of view angle
\item[right or wheel] adjust zoom factor
}
\end{Details}
\begin{SeeAlso}\relax
\code{\LinkA{par3d}{par3d}}
\end{SeeAlso}
\begin{Examples}
\begin{ExampleCode}

# animated round trip tour for 10 seconds

rgl.open()
shade3d(oh3d(), color="red")

start <- proc.time()[3]
while ((i <- 36*(proc.time()[3]-start)) < 360) {
  rgl.viewpoint(i,i/4); 
}

\end{ExampleCode}
\end{Examples}

\end{document}
